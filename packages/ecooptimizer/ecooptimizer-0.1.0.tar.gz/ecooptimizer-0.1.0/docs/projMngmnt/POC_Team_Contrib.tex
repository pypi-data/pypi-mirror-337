\documentclass{article}

\usepackage{float}
\restylefloat{table}

\usepackage{booktabs}

\title{Team Contributions: POC\\\progname}

\author{\authname}

\date{}

% Test

%% Comments

\usepackage{color}

% \newif\ifcomments\commentstrue %displays comments
\newif\ifcomments\commentsfalse %so that comments do not display

\ifcomments
\newcommand{\authornote}[3]{\textcolor{#1}{[#3 ---#2]}}
\newcommand{\todo}[1]{\textcolor{red}{[TODO: #1]}}
\else
\newcommand{\authornote}[3]{}
\newcommand{\todo}[1]{}
\fi

\newcommand{\wss}[1]{\authornote{blue}{SS}{#1}} 
\newcommand{\plt}[1]{\authornote{magenta}{TPLT}{#1}} %For explanation of the template
\newcommand{\an}[1]{\authornote{cyan}{Author}{#1}}

/* Copyright (c) 2021 by InterSystems Corporation.
   Cambridge, Massachusetts, U.S.A.  All rights reserved.
   Confidential property of InterSystems Corporation. */

Include Ensemble

Class IOP.Common [ Abstract, ClassType = "", ProcedureBlock, System = 4 ]
{

/// One or more Classpaths (separated by '|' character) needed in addition to the ones configured in the Java Gateway Service
Property %classpaths As %String(MAXLEN = "");

Property %classname As %String(MAXLEN = "");

Property %module As %String(MAXLEN = "");

Property %settings As %String(MAXLEN = "");

/// Instance of class
Property %class As %SYS.Python;

/// Get Class
Method GetClass() As %SYS.Python
{
    Return ..%class
}

/// Get Classname
Method GetClassname() As %String
{
    Return ..%classname
}

/// Get Classname
Method GetModule() As %String
{
    Return ..%module
}

Method %OnNew(pConfigName As %String) As %Status
{
    $$$ThrowOnError(..Connect())
	Quit $method($this,"initConfig",.pConfigName) ; call subclass
}

Method OnInit() As %Status
{
    set tSC = $$$OK
    try {
        
        do ..%class."_dispatch_on_init"($this)
    } catch ex {
        set tSC = ex.AsStatus()
    }
    quit tSC
}

ClassMethod SetPythonPath(pClasspaths)
{
    set sys = ##class(%SYS.Python).Import("sys")

    for i=0:1:(sys.path."__len__"()-1) {
        Try {
            if sys.path."__getitem__"(i) = pClasspaths {
                do sys.path."__delitem__"(i)
            }
        }
        Catch ex {
            // do nothing
        }

    }
    do sys.path.insert(0, pClasspaths)
}

Method Connect() As %Status
{
    set tSC = $$$OK
    try {
        
        set container = $this
        
        //set classpass
        if ..%classpaths '="" {
            set delimiter = $s($system.Version.GetOS()="Windows":";",1:":")
            set extraClasspaths = $tr(container.%classpaths,delimiter,"|")
            for i=1:1:$l(extraClasspaths,"|") {
                set onePath = $p(extraClasspaths,"|",i)
                set onePath = ##class(%File).NormalizeDirectory(onePath)
                do ..SetPythonPath(onePath)
            }
        }
        if $isObject(..%class)=0 {
            set importlib = ##class(%SYS.Python).Import("importlib")
            set builtins = ##class(%SYS.Python).Import("builtins")
            set module = importlib."import_module"(..%module)
            set class = builtins.getattr(module, ..%classname)
            set ..%class = class."__new__"(class)
        }
        ;
        if ..%Extends("IOP.InboundAdapter") || ..%Extends("IOP.OutboundAdapter") {
            do ..%class."_set_iris_handles"($this,..BusinessHost)
        } elseif $this.%Extends("IOP.BusinessProcess") {
            do ..%class."_set_iris_handles"($this,$$$NULLOREF)
        } else {
            do ..%class."_set_iris_handles"($this,..Adapter)
        }
        ;
        do ..SetPropertyValues()
        ;
        try {
            do ..%class."_dispatch_on_connected"($this)
        } catch ex {
            $$$LOGWARNING(ex.DisplayString())
        }
        ;
    } catch ex {
        set msg = $System.Status.GetOneStatusText(ex.AsStatus(),1)
        set tSC = $$$ERROR($$$EnsErrGeneral,msg)
    }
    quit tSC
}

Method OnTearDown() As %Status
{
    set tSC = $$$OK
    if $isObject(..%class) {
        try {
            do ..%class."_dispatch_on_tear_down"()
        } catch ex {
            set tSC = ex.AsStatus()
        }
    }
    quit tSC
}

Method SetPropertyValues()
{
    set remoteSettings = $tr(..%settings,$c(13))
    for i=1:1:$l(remoteSettings,$c(10)) {
        set oneLine = $p(remoteSettings,$c(10),i)
        set property = $p(oneLine,"=",1) continue:property=""
        set value = $p(oneLine,"=",2,*)
        try {
            set $property(..%class,property) = value
        } catch ex {
            $$$LOGWARNING(ex.DisplayString())
        }
    }
    quit
}

Method dispatchSendRequestSync(
	pTarget,
	pRequest,
	timeout,
	pDescription) As %String
{
    set tSC = ..SendRequestSync(pTarget,pRequest,.objResponse,timeout,pDescription)
    if $$$ISERR(tSC) throw ##class(%Exception.StatusException).CreateFromStatus(tSC)
    quit $g(objResponse)
}

Method dispatchSendRequestSyncMultiple(
	pCallStructList As %List,
	pTimeout As %Numeric = -1) As %List
{
    set builtins = ##class(%SYS.Python).Import("builtins")
    // Convert %List to multidimensional array
    set tCallStructList=builtins.len(pCallStructList)
    for i=0:1:builtins.len(pCallStructList)-1 {
        set tCallStructList(i+1) = pCallStructList."__getitem__"(i)
    }

    set tSC = ..SendRequestSyncMultiple(.tCallStructList,pTimeout)
    if $$$ISERR(tSC) throw ##class(%Exception.StatusException).CreateFromStatus(tSC)

    // Convert multidimensional array to Python list
    set tResponseList = builtins.list()
    
    for i=1:1:tCallStructList {
        do tResponseList.append(tCallStructList(i))
    }
    quit tResponseList
}

Method dispatchSendRequestAsync(
	pTarget,
	pRequest,
	pDescription)
{
    set tSC = ..SendRequestAsync(pTarget,pRequest,pDescription)
    if $$$ISERR(tSC) throw ##class(%Exception.StatusException).CreateFromStatus(tSC)
    quit
}

ClassMethod OnGetConnections(
	Output pArray As %String,
	pItem As Ens.Config.Item)
{
    // finds any settings of type Ens.DataType.ConfigName
    Try {
        do ..GetPropertyConnections(.pArray,pItem)
    }
    Catch ex {
    }

    // Get settings
    do pItem.GetModifiedSetting("%classpaths", .tClasspaths)
    do pItem.GetModifiedSetting("%classname", .tClassname)
    do pItem.GetModifiedSetting("%module", .tModule)

    // try to instantiate class
    if tClasspaths '="" {
            set sys = ##class(%SYS.Python).Import("sys")
            set delimiter = $s($system.Version.GetOS()="Windows":";",1:":")
            set extraClasspaths = $tr(tClasspaths,delimiter,"|")
            for i=1:1:$l(extraClasspaths,"|") {
                set onePath = $p(extraClasspaths,"|",i)
                set onePath = ##class(%File).NormalizeDirectory(onePath)
                if onePath?1"$$IRISHOME"1P.E set onePath = $e($system.Util.InstallDirectory(),1,*-1)_$e(onePath,11,*)
                if onePath'="" do sys.path.append(onePath)
            }
    }
    set importlib = ##class(%SYS.Python).Import("importlib")
    set builtins = ##class(%SYS.Python).Import("builtins")
    set module = importlib."import_module"(tModule)
    set class = builtins.getattr(module, tClassname)
    set tClass = class."__new__"(class)

    set tPythonList = tClass."on_get_connections"()
    set tPythonListLen = tPythonList."__len__"()
    for i=0:1:(tPythonListLen-1) {
        set tPythonItem = tPythonList."__getitem__"(i)
        set pArray(tPythonItem) = ""
		#; set ^AALog(pItem.Name,tPythonItem) = ""
    }

    quit
}

Method dispatchSendRequestAsyncNG(
	pTarget,
	pRequest,
	pTimeout,
	pDescription,
	ByRef pMessageHeaderId,
	ByRef pQueueName,
	ByRef pEndTime) As %String
{
	set tSC=$$$OK, tResponse=$$$NULLOREF
	try {

        set tTargetDispatchName=pTarget
        set tTargetConfigName=$get($$$DispatchNameToConfigName(pTarget))
        if tTargetConfigName="" set tSC=$$$EnsError($$$EnsErrBusinessDispatchNameNotRegistered,tTargetDispatchName) quit
        set tTargetBusinessClass = $$$ConfigClassName(tTargetConfigName)
        set tINVOCATION=$classmethod(tTargetBusinessClass,"%GetParameter","INVOCATION")
        if (tINVOCATION'="Queue")&&(tINVOCATION'="InProc") set tSC=$$$ERROR($$$EnsErrParameterInvocationInvalid,tTargetBusinessClass) quit

        quit:$$$ISERR(tSC)
        ;
        set tStartTime=$zh
        set:pTimeout'=-1 tEndTime=$zh+pTimeout

        if tINVOCATION="InProc" {
            set tTimeout=$s(pTimeout=-1:-1,1:tEndTime-$zh)
            if (pTimeout'=-1)&&(tTimeout<0) quit
            set tSC=..SendRequestSync(tTargetConfigName,pRequest,.tResponse,tTimeout,pDescription)
            return tResponse
        } elseif tINVOCATION="Queue" {
            Set tSessionId=..%SessionId
            Set tSuperSession = ..%SuperSession
            Set tSC = ##class(Ens.MessageHeader).NewRequestMessage(.tRequestHeader,pRequest,.tSessionId,.tSuperSession) quit:$$$ISERR(tSC)
            Set ..%SessionId=tSessionId
            Set ..%SuperSession=tSuperSession
            Set tRequestHeader.SourceConfigName = ..%ConfigName
            Set tRequestHeader.TargetConfigName = tTargetConfigName
            Set tRequestHeader.SourceBusinessType = $$$ConfigBusinessType($$$DispatchNameToConfigName(..%ConfigName))
            Set tRequestHeader.TargetBusinessType = $$$ConfigBusinessType($$$DispatchNameToConfigName(tTargetConfigName))
            Set tRequestHeader.TargetQueueName = $$$getConfigQueueName($$$DispatchNameToConfigName(tTargetConfigName),..%SessionId)
            Set tRequestHeader.ReturnQueueName = $$$queueSyncCallQueueName
            Set tRequestHeader.BusinessProcessId = ""
            Set tRequestHeader.Priority = $$$eMessagePriorityAsync
            Set tRequestHeader.Description = pDescription
            Set tSC = ##class(Ens.Queue).Create($$$queueSyncCallQueueName) quit:$$$ISERR(tSC)
            Set tSC = ##class(Ens.Queue).EnQueue(tRequestHeader) quit:$$$ISERR(tSC)
            Set pMessageHeaderId = tRequestHeader.MessageId()
            Set pQueueName = $$$queueSyncCallQueueName
            Set:(pTimeout'=-1) pEndTime = tEndTime
        }
	}
	catch {
		set tSC = $$$EnsSystemError
	}
	quit tSC
}

Method dispatchIsRequestDone(
	pTimeout,
	pEndTime,
	pQueueName,
	pMessageHeaderId,
	ByRef pResponse) As %Status
{

    set tSC=$$$OK
    try {
        set tTimeout=$s(pTimeout=-1:-1,1:pEndTime-$zh)

        set tSC = ##class(Ens.Queue).DeQueue($$$queueSyncCallQueueName,.tResponseHeader,tTimeout,.tIsTimedOut,0) Quit:$$$ISERR(tSC)

        quit:$IsObject(tResponseHeader)=0

        set tFound = $select(tResponseHeader.CorrespondingMessageId: pMessageHeaderId=tResponseHeader.CorrespondingMessageId, 1: 0)
        if tFound=0 {

            set tSC = ##class(Ens.Queue).EnQueue(tResponseHeader)
            Kill $$$EnsActiveMessage($$$SystemName_":"_$Job)
        }
        else {

            if tIsTimedOut || ((pTimeout'=-1)&&(tTimeout<0)) {

                do tResponseHeader.SetStatus($$$eMessageStatusDiscarded)
                return $$$ERROR($$$EnsErrFailureTimeout, tTimeout, $$$StatusDisplayString(tSC), $$$CurrentClass)
            }
            if tResponseHeader.IsError {

                do tResponseHeader.SetStatus($$$eMessageStatusCompleted)
                return $$$EnsError($$$EnsErrGeneral,"Error message received: "_tResponseHeader.ErrorText)
                
            }
            if tResponseHeader.MessageBodyClassName'="" {

                set tResponse = $classmethod(tResponseHeader.MessageBodyClassName,"%OpenId",tResponseHeader.MessageBodyId,,.tSC)
                if '$IsObject(tResponse) return $$$EnsError($$$EnsErrGeneral,"Could not open MessageBody "_tResponseHeader.MessageBodyId_" for MessageHeader #"_tResponseHeader.%Id()_" with body class "_tResponseHeader.MessageBodyClassName_":"_$$$StatusDisplayString(tSC)) 
            } else {

                set tResponse=$$$NULLOREF
            }
            set pResponse=tResponse
            do tResponseHeader.SetStatus($$$eMessageStatusCompleted)
            set tSC = 2

        }
    }
	catch ex {
		set tSC = ex.AsStatus()
	}
	quit tSC
}

}


\begin{document}

\maketitle

This document summarizes the contributions of each team member up to the POC
Demo.  The time period of interest is the time between the beginning of the term
and the POC demo.

\section{Demo Plans}

For our proof of concept demonstration, we will showcase the core functionality of our energy-efficient 
Python code refactoring tool. The demonstration will focus on the following key aspects:
\begin{enumerate}

    \item \textbf{Code Smell Detection:} We will show how we used Pylint to identify inefficient code 
    patterns (code smells) in Python source code that may lead to higher energy consumption.
    \item \textbf{Refactoring:} Using the Rope library, we'll demonstrate how our tool will 
    apply refactorings to address the detected code smells.
    \item \textbf{Energy Consumption Measurement:} We will show how we utilized CodeCarbon to measure 
    and compare the energy consumption of the original code versus the refactored version.
    \item \textbf{Functionality Preservation:} We will demonstrate that the refactored code 
    maintains its original functionality by running the original test suite against both versions of the code.
    \item \textbf{Performance Metrics:} We will display performance reports comparing the original and refactored 
    code, highlighting improvements in energy efficiency.
    
\end{enumerate}

\section{Team Meeting Attendance}

\begin{table}[H]
\centering
\begin{tabular}{ll}
\toprule
\textbf{Student} & \textbf{Meetings}\\
\midrule
Total & 11\\
Sevhena Walker & 11\\
Nivetha Kuruparan & 11\\
Tanveer Brar & 11\\
Mya Hussain & 11\\
Ayushi Amin & 11\\
\bottomrule
\end{tabular}
\end{table}

We aim to have all team members present for any team meetings we hold. If one person cannot attend the meeting, we usually reschedule to a different time.

\section{Supervisor/Stakeholder Meeting Attendance}

\begin{table}[H]
\centering
\begin{tabular}{ll}
\toprule
\textbf{Student} & \textbf{Meetings}\\
\midrule
Total & 4\\
Sevhena Walker & 4\\
Nivetha Kuruparan & 4\\
Tanveer Brar & 4\\
Mya Hussain & 4\\
Ayushi Amin & 4\\
\bottomrule
\end{tabular}
\end{table}

We aim to have all team members present for any meetings we have with our supervisor. If one person cannot attend the meeting, we usually reschedule to a different time.

\section{Lecture Attendance}

\begin{table}[H]
\centering
\begin{tabular}{ll}
\toprule
\textbf{Student} & \textbf{Lectures}\\
\midrule
Total & 12\\
Sevhena Walker & 11\\
Nivetha Kuruparan & 8\\
Tanveer Brar & 8\\
Mya Hussain & 6\\
Ayushi Amin & 6\\
\bottomrule
\end{tabular}
\end{table}

We aim to have at least one team member present for lectures to make sure we don't miss any critical information regarding deliverables.

\section{TA Document Discussion Attendance}

\begin{table}[H]
\centering
\begin{tabular}{ll}
\toprule
\textbf{Student} & \textbf{Lectures}\\
\midrule
Total & 3\\
Sevhena Walker & 3\\
Nivetha Kuruparan & 3\\
Tanveer Brar & 3\\
Mya Hussain & 3\\
Ayushi Amin & 3\\
\bottomrule
\end{tabular}
\end{table}

\section{Commits}

\begin{table}[H]
\centering
\begin{tabular}{lll}
\toprule
\textbf{Student} & \textbf{Commits} & \textbf{Percent}\\
\midrule
Total & 396 & 100\% \\
Ayushi Amin & 89 & 23\% \\
Tanveer Brar & 44 & 11\% \\
Mya Hussain & 59 & 15\% \\
Sevhena Walker & 156 & 39\% \\
Nivetha Kuruparan & 48 & 12\% \\
\bottomrule
\end{tabular}
\end{table}

Some people might have higher commit counts because they commit more frequently or make smaller, more granular changes, whereas others might commit less often with larger, consolidated changes. Additionally some group members squash and merge when merging PRs and others forget sometimes.

\section{Issue Tracker}

\wss{For each team member how many issues have they authored (including open and
closed issues (O+C)) and how many have they been assigned (only counting closed
issues (C only)) over the time period of interest.}

\begin{table}[H]
\centering
\begin{tabular}{lll}
\toprule
\textbf{Student} & \textbf{Authored (O+C)} & \textbf{Assigned (C only)}\\
\midrule
Sevhena Walker & 25 & 24 \\
Mya Hussain & 10 & 19 \\
Nivetha Kurparan & 18 & 23 \\
Tanveer Brar & 9 & 20 \\
Ayushi Amin & 12 & 21 \\
\bottomrule
\end{tabular}
\end{table}

The numbers in the \textbf{Assigned} column give a better picture of each team members contribution. Many commits were sometimes authored by the same person due to differences in team responsibilities (logistics and management). Furthermore, the issues here refer to what we can call ``work'' issues. Issues with the \texttt{lecture}, \texttt{team-meeting}, \texttt{sup-meeting}, and \texttt{ta-meeting} labels are not included in this tally.

\section{CICD}

The section outlines the plan to include CI/CD for this project. The plan will streamline development, testing and deployment processes, while ensuring consistent performance improvements.

\subsection{Source Control and Branching Strategy}
\begin{itemize}
    \item \textbf{Repository Setup}: Code is hosted on GitHub for version control and collaboration.
    \item \textbf{Branching Strategy}:
    \begin{itemize}
        \item \texttt{main}: Production-ready code.
        \item \texttt{dev}: Primary development branch.
        \item \texttt{docs}: Feature branch of dev that is meant for documentation commits.
    \end{itemize}
    Based on deliverables, temporary branches are created on team and individual level and discarded once merged into one of the above branches.
    In future, \texttt{dev} will be diverged into multiple feature branches for initial commits that are eventually merged into it. These include component specific branches such as \texttt{refactoring}, \texttt{analyser}, \texttt{energy}, \texttt{test} and \texttt{plugin}.
    \item \textbf{Merging Policy}: All pull requests should have at least two reviews before merging, as outlined in the Development Plan.
\end{itemize}

\subsection{Build and Testing Pipeline}
GitHub Actions will be used for CI/CD to automate testing and code analysis on pull requests. They will include the following:
\begin{itemize}
    \item \textbf{Build Steps}
    \begin{itemize}
        \item \textbf{Static Code Analysis \& Linting}: \texttt{PyLint} will be used to handle both code smells for static analysis and enforce PEP 8 style guide.
    \end{itemize}
    \item \textbf{Testing}:
    \begin{itemize}
        \item \textbf{Unit Tests}: Unit tests will be written using \texttt{PyTest}.
        \item \textbf{Code Coverage}: Test code coverage will be tracked using \texttt{coverage.py}.
        \item \textbf{Performance Testing}: Metrics such as memory usage and execution time will be tracked using \texttt{cProfile}.
    \end{itemize}
\end{itemize}

\subsection{Continuous Deployment}
With every stable version, the product will need to be continuously deployed.
\begin{itemize}
    \item \textbf{Environment Setup}: To standardize environment settings across platforms, Docker containers will be used.
    \item \textbf{Deployment}:
    \begin{itemize}
        \item \textbf{Refactoring Library}: The library will be rebuilt and updated on its public facing source.
        \item \textbf{VS Code Extension}: With each update to main branch, the VS Code extension will automatically be built and updated on its public facing link.
    \end{itemize}
\end{itemize}

\section{Additional Productivity Metrics}
The team does not have any additional metrics of productivity.
\end{document}