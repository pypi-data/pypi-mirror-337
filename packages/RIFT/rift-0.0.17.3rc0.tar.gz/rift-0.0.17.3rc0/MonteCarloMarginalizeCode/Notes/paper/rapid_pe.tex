\documentclass[twocolumn,prd,nofootinbib]{revtex4}
%\newcommand\ForRichardOnly[1]{#1}
\newcommand\ForRichardOnly[1]{}
\usepackage{verbatim}
\usepackage{color}     % color text
\usepackage{framed}
\definecolor{shadecolor}{gray}{0.95}
\usepackage{amsmath}
\usepackage{graphicx}  % extend graphics
\usepackage{tabularx}
\usepackage{wrapfig}   % wrap text around figures, if desired
\usepackage{hyperref}

%\newcommand\ForInternalReference[1]{#1}
\newcommand\ForInternalReference[1]{}
\newcommand\editremark[1]{{\color{red} #1}}
\usepackage{mathtools}
\DeclarePairedDelimiter\ceil{\lceil}{\rceil}
\DeclarePairedDelimiter\floor{\lfloor}{\rfloor}

\newcommand\unit[1]{{\rm #1}}
\newcommand\Y[1]{Y^{(#1)}{}}
% GR tools
\newcommand\prederiv[2]{{}^{(#1)}#2}
\newcommand\dualBack{*}
\newcommand\dualForward{\star}
\newcommand\avL{\left< {\cal L}_{(a} {\cal L}_{b)} \right>}
\newcommand\WeylScalar{{\psi_4}}
\newcommand\WeylScalarFourier{{\tilde{\psi}_4}}
\newcommand\mc{{{\cal M}_c}}
% QM TOOLS
\newcommand\qmstate[1]{\left|#1\right \rangle}
\newcommand\qmstateKet[1]{\left\langle#1\right|}
\newcommand\qmstateproduct[2]{\left\langle#1|#2\right\rangle}
\newcommand\qmoperatorelement[3]{\left\langle#1\left|#2\right|#3\right\rangle}
\newcommand\qmoperator[1]{{\bf #1}}


% NUMBERS
\newcommand\nEventsMDC{450}
\newcommand\BS{{\sc Bayestar}}
\newcommand\gstlal{{\sc GSTlal}}
\newcommand\Like{{\cal L}}
\newcommand\LikeRed{{{\cal L}_{\rm red}}}

% CITATIONS
\newcommand\citeMCMC{\cite{LIGO-CBC-S6-PE,2011PhRvD..83h2002D,2011PhRvD..84f2003C,gr-extensions-tests-Europeans2011,gwastro-mergers-PE-Aylott-LIGOATest,2011ApJ...739...99N,2012PhRvD..85j4045V,gw-astro-PE-Raymond,gw-astro-PE-lalinference-v1}}
\newcommand\citeFisher{\cite{1995PhRvD..52..848P,2008PhRvD..77d2001V,2008CQGra..25r4007C,2010PhRvD..82l4065V,gwastro-mergers-HeeSuk-FisherMatrixWithAmplitudeCorrections}}

% JOURNALS
\newcommand\pasp{PASP}
\newcommand\araa{ARAA}
\newcommand\mnras{MNRAS}
\newcommand\physrep{Phys. Rep.}
\newcommand\cqg{CQG}

% 
\newcommand\itrprm{\vec{\lambda}}
\newcommand\etrprm{\vec{\theta}}


\begin{document}

\title{Rapid, embarassingly parallel parameter estimation of gravitational waves from compact binary coalescences}
\author{P. Brady}
\email{patrick@gravity.phys.uwm.edu}
\affiliation{Center for Gravitation and Cosmology, University of Wisconsin-Milwaukee, Milwaukee, WI 53201, USA }
\author{E. Ochsner}
\email{evano@gravity.phys.uwm.edu}
\affiliation{Center for Gravitation and Cosmology, University of Wisconsin-Milwaukee, Milwaukee, WI 53201, USA }
\author{R. O'Shaughnessy}
\email{oshaughn@gravity.phys.uwm.edu}
\affiliation{Center for Gravitation and Cosmology, University of Wisconsin-Milwaukee, Milwaukee, WI 53201, USA }
\author{C. Pankow}
\email{pankow@gravity.phys.uwm.edu}
\affiliation{Center for Gravitation and Cosmology, University of Wisconsin-Milwaukee, Milwaukee, WI 53201, USA }

\begin{abstract}
We  introduce an alternative, highly-parallelizable architecture for compact binary parameter estimation.   
%   
First, by using a mode decomposition  ($h_{lm}$) to represent each physically distinct source and by
prefiltering the data against those modes, we can efficiently evaluate the likelihood for generic source positions and
orientations, independent of waveform length or generation time.   
% 
Second, by integrating over all observer-dependent (extrinsic) parameters and by using a purely Monte Carlo
integration strategy, we can efficiently \emph{parallelize} our calculation over the intrinsic and extrinsic space.  
%
Third, we use existing gravitational-wave searches to identify specific intrinsic (and extrinsic) parameters for further
investigation.  
% POINT: Conclusion: Code is fast
As  proof of concept, we have implemented this algorithm using standard time-domain waveforms in a
production environment, processing  each event in less than 1 hour, using roughly $1000$ cores in parallel,
producing posteriors and evidence with reproducibly small statistical errors (i.e., $\lesssim 1\%$ for both).   
%
%
As our code has bounded runtime, almost independent of the starting frequency for the signal, a nearly-unchanged strategy could
 estimate NS-NS parameters in the aLIGO era.  
% POINT: Mention *operational* for EOBNRv2HM
Our algorithm is both ready-to-use and efficient for any noise curve and existing time-domain model at any mass, including
slow-to-evaluate waveforms like EOBNRv2HM.  
\end{abstract}
\maketitle

\begin{widetext}
{\color{red}\textbf{Draft, not for distribution except by permission of the authors}}

{\color{blue}\textbf{For other authors}: Enable internal-use info (development plans; future project ideas; feedback notes) via changing the
  definition of \texttt{\\ForInternalReference}}
\end{widetext}


\ForInternalReference{
\begin{widetext}
\section*{Outline}
The first level of bullets are proposed sections. 
The second level of bullets (except in Executive Summary) are proposed subsections.
The third level of bullets are points to make in each subsection.

Section headings added, including reminders re dotting i's/crossing t's.

\begin{itemize}
\item Executive Summary
	\begin{itemize}
	\item Goal: Rapid parameter estimation of CBC GW signals -- target is a few minutes!
	\item Trick \# 1: use mode decomposition and compute $( h_{\ell m} | d )$ once for each mass pair.
		Extremely fast likelihood evaluation as extrinsic parameters are varied.
	\item Trick \# 2: Abandon Markov chain/ nested sampling. They have nice convergence in high dimensions,
		but are serial. Instead, use brute force grid and Monte Carlo technique. Convergence worse in a sense,
		but embarassingly parallel and very rapid evaluations thanks to trick \# 1.
	\item Essentially same cost regardless of waveform evaluation speed. Therefore can use very long signals 
		and/or expensive models like EOB. 
	\end{itemize}

\item Methods
	\begin{itemize}
        \item Background
           
	\item Likelihood evaluation
		\begin{itemize}
		\item Start with expression for ${\cal L}$. 
		\item Show steps to get in terms of $( h_{\ell m} | d )$.
		\item Note how all extrinsic parameters enter in $Y_{\ell m}$'s and $F_+$, $F_\times$, and thus we can
			evaluate likelihood cheaply for any extrinsic parameters if we fix the intrinsic ones.
		\end{itemize}
		
		
	\item Integration over extrinsic parameters
		\begin{itemize}
		\item Do a basic Monte Carlo integral over  extrinsic parameters, at fixed intrinsic parameters
                        Priors (e.g., time window)
%		\item Describe any fancy pants adaption, use of skymaps, etc. 
		\end{itemize}
                \begin{itemize}
  	         \item Time marginalization  \textbf{subsection of extrinsic} 
		  \begin{itemize}
		  \item Inverse FFT trick gives $(h_{\ell m}(t_c) | d)$ for all values of $t_c$ at once.
		  \item We just sum over a reasonably sized window $\sim 10$ ms. 
		  \end{itemize}
                 \item Adaptation (in distance)
                 \item Targeted sampling : skymaps
                \end{itemize}

	\item Covering the intrinsic parameters
		\begin{itemize}
		\item Describe effective Fisher approach to laying out mass points.
		\item Point out quite flexible: can do random or fixed grid, can change ellipse size, distribution inside ellipse,
			could do several ellipses centered on different points. [?]
		\item Point out difficult to go to many intrinsic dimensions. Right now we focus on 2D non-spinning, 
			but 3D or 4D is possibly feasible.
			Precession would be very tough. 
			Maybe an approximate metric or help from ROM could save the day. [?]
		\end{itemize}
		
	
	\item Postprocessing [?]
		\begin{itemize}
		\item Do we need to say anything about collating results, making triplots, P-P plots, etc.?

                  Yes, it's nontrivial for linear spoked.
		\end{itemize}

	\end{itemize}

\item Results: Production environment
	\begin{itemize}
          \item Describe the BNS MDC, from which our examples are drawn
	\item Show posteriors, convince reader they're similar enough to usual Bayesian PE to be trusted.
	\item P-P plots for an ensemble of physical injections (including spin). Show our results are self-consistent.

          Malmquist/selection bias.  
	\item Brief (not more than 1-2 paragraphs, 1 simple plot/table) results confirming 
		that our method scales favorably with $f_{\rm min}$ and/or EOB. 
	\end{itemize}

\item Conclusions -- self-explanatory.

\item Appendices -- Put detailed technical asides here.

  \begin{itemize}
    \item Notation
    \item Data handling: data rate and data selection (time window), psd estimate, inverse spectrum truncation,
      windowing (or not)
  \item Results: Targeted investigations

  -    DAGs on a single event

  - DAGs using the same physics (zero spin) as our model


  \item General and adaptive monte carlo: For-future-reference stuff, including

    - uniform and variance-minimizing weighting to combine integral results with different samplers

  \end{itemize}

\end{itemize}

Citations: Bayesian methods and monte carlo integration \cite{2011RvMP...83..943V}, including numerical recipes and Lepage; comparison
to other PE methods for LIGO
(MCMC,nested sampling) \cite{LIGO-CBC-S6-PE,2011PhRvD..83h2002D,2011PhRvD..84f2003C,gr-extensions-tests-Europeans2011,gwastro-mergers-PE-Aylott-LIGOATest,2011ApJ...739...99N,2012PhRvD..85j4045V,gw-astro-PE-Raymond,gw-astro-PE-lalinference-v1}

\tableofcontents

\end{widetext}
}
\section{Introduction}
\label{sec:introduction}

Ground based gravitational wave detector networks (notably LIGO \cite{gw-detectors-LIGO-original-preferred} and Virgo
\cite{gw-detectors-VIRGO-original-preferred})  are sensitive to the gravitational wave signal from coalescing compact
binaries, both the relatively well understood signal from  the lowest-mass compact binaries
$M=m_1+m_2\le 16 M_\odot$
\cite{2003PhRvD..67j4025B,2004PhRvD..70j4003B,2004PhRvD..70f4028D,BCV:PTF,2005PhRvD..71b4039K,2005PhRvD..72h4027B,2006PhRvD..73l4012K,2007MNRAS.374..721T,2008PhRvD..78j4007H,gr-astro-eccentric-NR-2008,gw-astro-mergers-approximations-SpinningPNHigherHarmonics,gw-astro-PN-Comparison-AlessandraSathya2009}
and the less-well-understood strong-field epoch
\cite{2011PhRvD..83l2005A,2009CQGra..26p5008A, 2014PhRvD..89d2002K,2009PhRvD..79l4028B,2010PhRvD..82f4016S,2011CQGra..28m4002M,2013PhRvD..87b4009M,2013CQGra..31b5012H}.    
%
% POINT: Inference via Bayesian methods
Interpreting gravitational wave data requires systematically comparing all possible candidate signals to the data,
constructing a Bayesian posterior probability distribution for candidate binary parameters \citeMCMC{}.   
%
Owing to the complexity and multimodality of the model space, historically successful strategies have adopted  generic, serial
algorithms for parameter estimation, such as variants of Markov Chain Monte Carlo or nested sampling
\cite{2011RvMP...83..943V,gw-astro-PE-lalinference-v1}.   
%
In physics, similar path-based methods have been enormously successful at a broad range of physical problems, by
exploring all  possible paths  through a configuration space; see,
e.g.,  \cite{2001RvMP...73...33F,1987PhLB..195..216D}.  
% POINT: Why do something new?  
%   -Because they are slow, being serial
Though successful, these generic algorithms are functionally serial, requiring intensive communication to coordinate the
current state, and are therefore well-known to scale poorly to large numbers of processors.  
%
These algorithms' convergence is also limited by \emph{ergodicity}.  
 No theorem  guarantees these algorithm \emph{must}  explore the entire model space, let alone efficiently; no expression can robustly assess convergence, using available
 sampled data.  
%
By contrast, well-understood, efficient, and highly-parallelizable Monte Carlo integration strategies are frequently applied to problems with dimensions comparable or
higher than coalescing compact binaries; see, e.g.,  \cite{lepage1980vegas,1980PhRvD..21.2308C,book-math-Jaeckel-MonteCarlo,mm-QuasiMonteCarlo-Papageorgiou2001}.    In this work, we apply such methods to gravitational wave
parameter estimation for the first time.  
%
% http://en.wikipedia.org/wiki/Quasi-Monte_Carlo_methods_in_finance
%
%  Beware re quantum monte carlo -- often it is basically an MCMC
%    - http://adsabs.harvard.edu/abs/2001RvMP...73...33F
%    - 1987PhLB..195..216D,


% POINT :
%  - because they are slow, not communicating 
Gravitational wave parameter estimation has also been limited by the computational cost associated with comparing the
data with a candidate waveform.  This cost has two factors: first, the cost of waveform generation, as discussed in
\cite{gwastro-mergers-PE-ReducedOrder-2013,2013PhRvD..87l2002S,2013PhRvD..87d4008C,gwastro-mergers-IMRPhenomP,gwastro-SpinTaylorF2-2013} and references therein; and, second,
the cost of repeatedly operating on the long time- or frequency-series itself, once per comparison (e.g., fourier
transforms).   
%
Recently, several methods have been proposed to perform this comparison more efficiently
\cite{gwastro-mergers-PE-ReducedOrder-2013,2013PhRvD..87l2002S,2013PhRvD..87d4008C,gw-astro-ReducedOrderQuadraturePE-TiglioEtAl2014}, by interpolating some combination
of the the waveform or likelihood or by adopting a sparse representation to reduce the computational cost of data
handling.  
%
In this work, we introduce a new, simple, and extremely robust scheme to reduce the cost per comparison.  
%
To our knowledge, this work is the first time any such method has been implemented at production scale, particularly for
the most accurate and computationally-expensive waveform models like EOB
\cite{gw-astro-EOBspin-Tarrachini2012,gw-astro-EOBNR-Calibrated-2009}.  


% POINT: Blindly redo the anaysis, versus using information from search?
Historically, parameter estimation strategies very practically make almost no use of the information reported by
existing search codes.  This careful approach ensured, for example, that inferences from the data would never be used as priors on a
reanalysis of the same data.  
%
That said, experience suggests the search codes provide a reasonable first approximation, 
particularly for the most tightly constrained parameters, relative to their priors (i.e., event time; chirp time; sky location).  
%
Some parameter estimation codes have used this information to improve their performance, more efficiently searching the
sky \cite{2013APS..APRG10003F,2013arXiv1309.7709F}. % \editremark{check with Ben F}.  
That said, historical experience with serial codes suggest \emph{sampling} the maximum (not finding it) is what limits
performance; based on that experience, one might expect minimal additional asymptotic speed improvement when provided
with a good initial guess.
%
By contrast, Monte Carlo integration trivially provides robust sampling of any weakly-constrained parameters (e.g.,
distance, inclination, polarization).    By using search results to sample near expected results, Monte Carlo methods
can be both generic and extremely efficient.   We therefore describe a procedure systematically and organically incorporates information provided
by the search to accelerate our specific algorithm's performance.  
%
Combining these three factors (parallelizable monte carlo integration; efficient likelihoods; and search information),
we can provide reliable parameter estimates for merging double neutron star binaries within one hour on existing
hardware for instruments available in the next two years.


% POINT: Low latency as ratioale
In the era of multimessenger astronomy, rapid and robust inference about candidate compact binary gravitational wave
events will be a critical science product for LIGO, as  colleagues with other instruments  perform followup and
coincident observations \cite{LIGO-2013-WhitePaper-CoordinatedEMObserving}.  
%
The most tantalizing proposed electromagnetic counterparts are expected to be brief, potentially disappearing within
days if not much
sooner \cite{2012ApJ...746...48M,2014MNRAS.439..757G,2014MNRAS.437L...6K,2014MNRAS.437.1821M,2014ApJ...780...31T,2013ApJ...775...18B}.  
%
Given limited resources, reliable low-latency parameter estimation of gravitational wave signals will significantly enhance the science output of
multimessener, time-domain astronomy.  

% ``burnin is short; finding the best fit is fast;
%what's expensive is mapping out the likelihood near the maximum'' \editremark{fixme: can't use in publication}.
% 


% POINT: Outline
This work is organized as follows.
In Section \ref{sec:Executive} we provide an executive summary, providing the principles that enable our algorithm to
provide rapid, accurate results for the test cases explored here:  time-domain models for nonprecessing binaries.  
%
Then, in Section \ref{sec:Methods}, we describe our algorithm in detail.  
%
To demonstrate  our algorithm provides high performance in a production environment,  Section \ref{sec:Results} we
provide concrete results drawn from a large sample of events from the ``2015 double neutron star mock data challenge'',
results from which will be described in \cite{first2years}.  
%
In Section \ref{sec:Discussion} we reflect on the broader significance of our result, in the context of other parameter
estimation work inside and outside the LSC.  
%
We conclude with Section \ref{sec:Conclude}.  


\section{Executive Summary}
\label{sec:Executive}
\subsection{Motivation}

% low latency searches -- both GW and EM
The scheduled resumption of data taking in late 2015\cite{LIGO-2013-WhitePaper-CoordinatedEMObserving}, with the second generation gravitational-wave interferometers in Hanford, Washington, and Livingston, Louisiana, are expected to reach sensitivities which are incrementally better than those achieved in earlier data taking runs\cite{s5s6}. The Virgo detector is also expected to resume data taking within a year of this milestone. In preparation for the next run, the LIGO and Virgo Collaborations have implemented and extensively tested a set of low latency gravitational-wave detection pipelines\cite{gstlal,mbta,cwb}, capable of compact binary event detection in latencies of a few minutes or less from the coalescence time. These pipelines trade the ability to accurately determine all but a few of the parameters of the coalescence for speed and breadth of analysis. Having a more accurate estimation of the parameters is valuable not only to gravitational-wave science, but also is useful to provide electromagnetic observatories information to aid in pointing. Many astrophysical phenomena which could create transient gravitational waves will also have electromagnetic signatures that may rapidly decay, and gravitational-wave interferometer networks often cannot localize GW events to better than a few hundred square degrees on the sky\cite{LIGO-2013-WhitePaper-CoordinatedEMObserving,first2years,gwastro-skyloc-Sidery2013,skylocpapers}, thus prompting follow-up observations occur expeditiously after the initial identification as well as localized promptly and accurately. While several Bayesian algorithms for GW parameter estimation have been employed in the past, the time scale of a full parameter estimation analysis remains at a much higher latency than the initial detection.

% MC integration scheme
%Given the inherent serial nature of current Bayesian parameter estimation schemes,  s
Significant reductions in overall computational time seem to  require  reduced the cost per likelihood computation or an
inherently parallel  parallel algorithm from the start. Our scheme accomplishes both.  
%
On the one hand, our scheme precomputes several quantities which appear in the likelihood, allowing us to  efficiently and
repeatedly evaluate it with little cost.  By contrast, most current schemes  require the costly evaluation of an inner
product per likelihood sample.  
%
%
On the other hand, our scheme parallelizes over intrinsic parameters (i.e., masses).    Because all waveforms are
generated once, in parallel, we mitigate the cost of waveform generation, to the point that even waveform families with
significant computational overhead per waveform are no serious obstacle.\footnote{Roughly speaking, more accurate and
complete physical models are more expensive to generate. Several waveform families representing the gravitational radiation from coalescing binaries exist and are in common use,
both for searches and followup parameter
estimation\cite{gw-astro-mergers-approximations-SpinningPNHigherHarmonics,gw-astro-PN-Comparison-AlessandraSathya2009,gw-astro-EOBspin-Tarrachini2012,gw-astro-EOBNR-Calibrated-2009}.}  This method is particularly beneficial as 
the low frequency sensitivity of the second generation interferometers evolves, requiring  waveform simulations to start
at lower frequencies and cover a significantly longer time interval, at correspondingly larger computational cost. 
%
By contrast, current parameter estimation schemes usually require millions of likelihood evaluations
to explore the entire space, with a correspondingly large number of  waveform generations.   Waveform generation cost
can dominate the overall cost of the calculation.

% waveform generation cost

% POINT: Not only accurate
Because we can direct evaluate the expected accuracy in each Monte Carlo integral, using the observed variance we can
insure our algorithm terminates only once  a target accuracy has been achieved.   Our robust error estimates are  vastly less complex than comparable algorithms to derive the evidence from MCMC\cite{gwastro-mergers-HeeSuk-CompareToPE-Aligned,mm-stats-MCMC-GeometricLaddear-Neal,mm-stats-MCMC-GeometricLadderChoices-Liu,mm-stats-EvidenceFromMCMC-Weinberg2009,2007PhRvD..75f2004R,2008ApJ...688L..61V,2009PhRvD..80f3007L, 2009CQGra..26k4007R,2011RvMP...83..943V}.

We also propose to make use of information that is retrieved from both the low-latency gravitational-wave search as well as any other low-latency process to augment our own results. From the gravitational-wave search, we obtain knowledge of the event time --- needed only in a general sense to restrict the amount of data processing involved to tractable levels --- and information regarding the masses, used later to center the search space in the searched intrinsic parameters. Fast algorithms which can produce an accurate estimate of the sky position using only the information provided from the GW search\cite{gw-astro-Bayestar} can also be utilized to speed up computation.

%% % TODO: Can we edit this down further?
%% {\color{blue} Evan thinks this is unclear and Chris agrees. Either edit it to be more concise or remove.}
%% Moreover, it is known\cite{glitch,detchar,glitch_pe} that the quality or stationarity of the data could influence the detection of an astrophysical event, or in the worst cases, trigger detection pipelines from the accidental coincidence of environmental or mechanical phenomenon which are not successfully cleaned from the data stream. The ability of a Bayesian parameter estimation algorithm to deal with such influences is an active area of study\cite{glitchfitting}. Even so, a detailed parameter estimation study can also influence the statistical significance of a putative event by downranking or upranking based on the model support from the parameter estimation itself.



\section{Binary Waveforms}
\subsection{Intrinsic and extrinsic parameters}

On physical grounds, we group waveform parameters $\vec{\mu}=(\vec{\lambda},\vec{\theta})$ into two classes: the
intrinsic parameters ($\vec{\lambda}$) and extrinsic parameters ($\vec{\theta}$). 
The intrinsic parameters are fundamental to the description of the binary: if we change any intrinsic
parameters we must recompute the orbital dynamics of the binary (typically through the relatively expensive process
of numerically integrating ordinary differential equations). Extrinsic parameters simply describe how the
binary is oriented in space and time relative to the detector; changing extrinsic parameters involves a 
relatively inexpensive rotation, translation or rescaling transformation. As we will show in the next subsection, 
for the non-spinning case considered here the intrinsic parameters are
\begin{equation} \label{eq:intrinsic}
\itrprm=\{\mc,\eta\}\ ,
\end{equation}
where $\mc = (m_1m_2)^{3/5}/(m_1+m_2)^{1/5}$) is the chirp mass and 
$\eta = m_1m_2/(m_1+m_2)^2$ is the symmetric mass ratio.
The extrinsic parameters are
\begin{equation} \label{eq:extrinsic}
\etrprm=\{t_{\rm geo},\alpha,\delta,\iota,D,\psi,\phi_c\}\ ,
\end{equation}
where $t_{\rm geo}$ is the time at which the waveform coalescence arrives at the Earth geocenter,
$\alpha$ and $\delta$ are the right ascension and declination, 
$\iota$ is the inclination angle of the binary's angular momentum vector and the line of sight to Earth, 
$D$ is the luminosity distance to the binary\footnote{For the sources considered in this paper, 
	the redshift correction is assumed to be negligible}, 
$\psi$ is the polarization angle, and $\phi_c$ is the orbital phase of the binary at coalescence.

Note that throughout the rest of the paper we will denote intrinsic parameters with $\itrprm$ 
and extrinsic parameters with $\etrprm$.

\subsection{Waveform decomposition}

The gravitational wave strain measured by the $k^{\rm th}$ interferometric detector in a network is given by
\begin{equation} \label{eq:measured_strain}
h_k(t) = F_{+,k}(\delta, \alpha, \psi) h_{+,k}(t) 
+ F_{\times,k}(\delta, \alpha, \psi) h_{\times,k}(t)\ ,
\end{equation}
where $F_{+,k}$, $F_{\times,k}$ are the antenna patterns of the detector and $h_{+,k}$, $h_{\times,k}$
are the two components of the gravitational wave strain, evaluated at the $k^{\rm th}$ detector. The antenna patterns depend only on the extrinsic sky location and polarization
angle, while the polarizations depend on both intrinsic and extrinsic parameters.   In turn, the polarizations can be
evaluated via the real and imaginary parts of a complex-valued plane wave $h(t|\mc,\eta,\iota,\phi_c)$:
\begin{align}
h_{+,k}(t)-i h_{\times, k}(t) &= h(t-t_k|\mc,\eta,\iota,\phi_c)
\end{align}

In this expression, $t_k$  denotes the time of arrival of the coalescence  at the $k^{th}$ detector,

\begin{equation} \label{eq:t_k}
t_k = t_{\rm geo} - \frac{\vec{x}_k \cdot \hat{N}(\alpha,\delta)}{c}\ .
\end{equation}

If $\vec{x}_k$ is a vector pointing from the geocenter to the $k^{th}$ detector and $\hat{N}(\alpha,\delta)$ is the direction of GW propagation, so each member of the network will have an offset relative to the geocenter time depending on sky location. 

At this stage, we make no assumptions about the functional form of $h(t|\mc,\eta,\iota,\phi_c)$
%
We can use a $-2$ spin-weighted spherical harmonic mode decomposition to further separate intrinsic and extrinsic
parameters appearing in the polarizations. 
In particular, we relate the polarizations to a set of $-2$ spin-weighted 
spherical harmonics modes with the equation
\begin{equation} \label{eq:h:Expansion}
h_{+,k}(t) - i\,h_{\times,k}(t) = \frac{D_{\rm ref}}{D} \sum_{lm} \hat{h}_{lm}(\mc,\eta,t_k;t)  \Y{-2}_{lm}(\iota,-\phi_c) \ ,
\end{equation}
For the purposes of this work, investigating radiation from inspiralling binaries, we will use $\hat{h}_{lm}$ as the -2
spin-weighted spherical harmonics modes provided by post-Newtonian calculations for adiabatic quasicircular inpiral \cite{gw-astro-mergers-approximations-SpinningPNHigherHarmonics},
evaluated at some fixed distance $D_{\rm ref}$ (in this work we choose $D_{\rm ref} = 100$ Mpc).
%
As a concrete example, at leading order for inspiral-only waveforms the polarizations are
\begin{widetext}
\begin{eqnarray} \label{eq:polarizations}
h_{+,k}(t) &=& - \frac{2 M \eta}{D}\,v^2(\mc,\eta,t_k;t)\,\left[ \left( 1 + \cos^2 \iota \right)\, \label{eq:h+}
	\cos 2 \left( \Phi(\mc,\eta,t_k;t) - \phi_c \right) + {\cal O}(v^3)\right]\ ,\\
h_{\times,k}(t) &=& - \frac{2 M \eta}{D}\,v^2(\mc,\eta,t_k;t)\,\left[ \left( 2 \cos \iota\right)\, \label{eq:hx}
	\sin 2 \left( \Phi(\mc,\eta,t_k;t) - \phi_c \right) + {\cal O}(v^3)\right]\ .
\end{eqnarray}
\end{widetext}
$\Phi(t)$ and $v(t)$ are the key orbital dynamical quantities, typically obtained via the energy balance equation,
which are expensive to compute.
%
If we define a complex-valued antenna pattern for each detector as
\begin{equation} \label{eq:complexF}
F_k = F_{+,k} + i \, F_{\times,k} \ ,
\end{equation}
then we can re-express the measured strain in the $k^{th}$ detector as
\begin{widetext}
\begin{equation} \label{eq:decomposed_strain}
h_k(\itrprm,\etrprm;t) = {\rm Re}\ \frac{D_{\rm ref}}{D}\, F_k(\alpha,\delta,\psi) \, \sum_{lm} 
\hat{h}_{lm}(\mc,\eta,t_k;t)  \Y{-2}_{lm}(\iota,-\phi_c)\ .
\end{equation}
\end{widetext}
Aside from $t_k$, we have now completely separated the intrinsic parameters (which enter only the $\hat{h}_{lm}$)
from the extrinsic parameters (which enter only the $F_k$ and $\Y{-2}_{lm}$).

Given a time-domain representation of the gravitational wave strain, we can define a 
frequency-domain version of this strain via a Fourier transform
\begin{equation} \label{eq:Fourier}
\tilde{h}(f) = \int_{-\infty}^\infty h(t) e^{- 2 \pi i f t} dt\ .
\end{equation}
Time translation of frequency-domain waveforms is trivial.
If $\tilde{h}(t_k;f)$ is the Fourier-domain representation of a strain for some arrival time $t_k$, then
the same strain arriving at another time $t_k'$ can be simply related by
\begin{equation} \label{eq:time_shift}
\tilde{h}(t_k';f) = \tilde{h}(t_k;f) \, e^{- 2 \pi i f (t_k' - t_k)}\ .
\end{equation}
Thus, if we work with frequency-domain waveforms, the arrival time $t_k$ 
can be factored out as $\exp( - 2 \pi i f t_k)$
and we can complete the separation of intrinsic and extrinsic parameters.

For the explicit decomposition above, we have focused on non-spinning waveforms 
and found that they can be separated into two intrinsic parameters and seven
extrinsic parameters. In the general case, which could include precession,
tidal effects and any other physics, this intrinsic and extrinsic separation is still
possible. In fact, there will always be seven extrinsic parameters which enter only
through inexpensive geometric factors, while any additional parameters will always
be encoded in the expensive $h_{lm}$ modes.

To see that this is true, first note that Eq.~(\ref{eq:time_shift}) holds for an arbitrary strain
and so can always be used to factor out the dependence on time of arrival. Similarly,
a gravitational wave far from its source will always fall off as $1/D$. Furthermore, 
the antenna patterns $F_+$ and $F_\times$ depend on the detector geometry,
not the source, and so are unchanged. This gives a total of five extrinsic parameters
that in general can be factored out exactly as in our non-spinning waveform. Lastly, 
we have two angles which enter into the $\Y{-2}_{lm}$. The physical interpretation 
of these angles depends on how the frame in which the harmonic mode decomposition is 
performed is defined. One should choose a frame which is convenient for expressing 
and computing the $h_{lm}$. For the non-spinning case, this means aligning the frame 
with the orbital angular momentum, $\hat{L}$, in which case one can show that the 
zenith angle is $\iota$ and the azimuth is $- \phi_c$. In the precessing case, one would
most likely use a frame aligned with the total angular momentum, $\hat{J}$, e.g. using the
parameterization described in~\cite{Farr:2014qka}. In the notation of that paper, the extrinsic
spherical harmonic dependence is $\Y{-2}_{lm}(\theta_{JN},-\phi_{JL})$, 
where $\theta_{JN}$ is the inclination of the \emph{total} angular momentum to line of sight,
and $\phi_{JL}$ marks the azimuthal position of $\hat{L}$ on its precession cone about $\hat{J}$.
Note that if we use such a frame $\iota$ and $\phi_c$ must be encoded in the $h_{lm}$, 
as must any other parameters such as spin vectors or tidal deformabilities.




\section{Methods}
\label{sec:Methods}

%* coordinates [limit to nonprecessing?]


%As summarized in Section \ref{sec:Executive}, our algorithm provides a parallel scheme to compare generic binary
%waveforms against data, using a low-cost likelihood function and information from the search to reduce computational cost and
%improve performance.  
%
%In this section, we provide the details needed to implement it.  
%
%Rather than dwell on relatively well-understood data handling issues,  we refer the reader to
%Appendix \ref{ap:DiscreteData} for further discussion of operations on discrete time- and frequency- series.

% POINT: Break up the 

A set of data $d(t)$ collected from a gravitational-wave detector is typically decomposed into the inherent noise of the detector and a putative gravitational wave signal:

\begin{equation}
d(t) = h(t) + n(t)\ .
\end{equation}

In the absence of a signal and under the assumption that each detector produces stationary Gaussian noise, the noise $\tilde{n}(f)$ in the $k^{\rm th}$ detector is characterized by its power spectrum $S_k(f)$:
\begin{eqnarray}
\left<\tilde{n}_k(f)^* \tilde{n}_k(f')\right> = \frac{1}{2} S_k(|f|) \delta(f-f')
\end{eqnarray}

In the following discussion, we define a weighted inner-product of two complex Fourier-domain functions $\tilde{a}(f), \tilde{b}(f)$ with a weighting function $1/S(f)$:

\begin{eqnarray} \label{eq:InnerProduct}
\qmstateproduct{a}{b} \equiv 2 \int_{-\infty}^{\infty} df \frac{\tilde{a}^*(f)\tilde{b}(f)}{S(f)}\
\end{eqnarray}

The mode decomposed waveforms $\hat{h}_{lm}$, can be shifted in time in reference 
to a detector relative to the geocentric time via Eq.~(\ref{eq:time_shift}). 
For this reason, the overlap between a single detector data time series  
$d(t)$ and a time-shifted complex function $h(t-t_k)$ is 
%% Plugging Eq.~(\ref{eq:time_shift}) into Eq.~(\ref{eq:InnerProduct}), the overlap between data $d$ and
%% some template $h(t_k)$ with time of arrival $t_k$ is
\begin{equation} \label{eq:IFFT}
\langle h(t_k) | d \rangle = \int_{-\infty}^{\infty} \frac{\tilde{d}(f) \tilde{h}^*(f)}{S_n(f)} e^{2\pi i f t_k} df \ .
\end{equation}
Note that this is simply the Fourier transform of the integrand in Eq.~(\ref{eq:InnerProduct}), 
which means we can compute the overlap for all possible time shifts with a single
Fourier transform.



%\subsection{Preliminaries}
%We begin with the division of the parameter space of compact binary coalescence waveforms. We first define the difference between ``intrinsic'' and ``extrinsic'' parameters. We define $\itrprm$ as the set of intrinsic parameter as those corresponding to the physical configuration of a binary with component masses $m_1$ and $m_2$: $\itrprm=\{\mc,\eta,\Lambda_1,\Lambda_2,\vec{S}_1,\vec{S}_2\}$, where $\mc$ and $\eta$ are the chirp mass ($(m_1m_2)^{3/5}/(m_1+m_2)^{1/5}$) and symmetric mass ratio ($m_1m_2/(m_1+m_2)^2$), $\Lambda_1$ and $\Lambda_2$ are the tidal numbers of each component mass\footnote{Black holes have no tidal number and thus $\Lambda=0$}, and finally $\vec{S}_1$ and $\vec{S}_2$ correspond to the spin vectors of the component compact objects\footnote{$\vec{L}$, the overall angular momentum of the system is important in the case where the spins are not aligned with the orbital angular momentum, thus causing precession of the orbital plane}. We emphasize in this work rapid determination of source masses --- and possibly compact object type --- thus focusing attention exclusively on $\itrprm=\{\mc,\eta\}$. While the impact of spin and tidal parameters on the waveform are important, they are also potentially complicate the computation of the likelihood in the scheme described later. We thus leave the remaining intrinsic parameters to be built upon in future work.
%
%The extrinsic parameters can be interpreted as the effect of the waveform's arrival and response on a given gravitational-wave detector. For a set of gravitational-wave interferometers, a seven-dimensional space of extrinsic parameters is defined as $\etrprm=\{t_r,\alpha,\delta,\iota,D,\psi,\phi\}$, where $t_r$ is a reference time (in this work, the geocentric time of coalescence), $\alpha$ and $\delta$ are the right ascension and declination, $\iota$ is the inclination angle of the binary's angular momentum vector and the line of sight to Earth, $D$ is the luminosity distance to the binary\footnote{For the sources considered in this paper, the redshift correction is assumed to be negligible}, $\psi$ is the azimuthal angle of the wave's plane of disturbance to the plane of the arms of the detector, and $\phi$ is the orbital phase of the binary at coalescence. Several subsets of these parameters have well-known (particularly in the non-spinning case) correlations.

\subsection{Bayesian evidence and posteriors}
%% We reconstruct the parameter

%% \textbf{Concept}* Top-level review: what are we doing (parameter estimation/posterior); establish notation
%% ($\lambda,\theta,L,p,\ldots$)
%% -----

Here we briefly summarize how Bayes' theorem can be used to quantify how likely it is the data 
(denoted by $\{d\}$) contains only Gaussian noise (hypothesis ${\cal H}_0$) 
or whether it consists of Gaussian noise plus a coherent GW signal in each of our instruments
(hypothesis ${\cal H}_1$). Additionally, it can be used to determine the likely parameters of a signal,
if present.

By our assumption that the noise is Gaussian, it follows that the probability of some set of noise
realizations in each of our detectors is given by
\begin{equation}
p(\{d\}|{\cal H}_0) \propto \prod_k \exp - \frac{ \langle d | d \rangle_k}{2}\ .
\end{equation}
So the data in each detector is noise plus a GW signal 
with true parameters $\vec{\mu}_0$, then the likelihood of some parameters values $\vec{\mu}$ is given by
\begin{equation}
p(\{d\}|\vec{\mu},{\cal H}_1) \propto \prod_k \exp - \frac{ \langle d - h(\vec{\mu}) | d - h(\vec{\mu}) \rangle_k }{2}\ .
\end{equation}
In essence, we subtract a waveform with parameters $\vec{\mu}$ from the data 
and compute how consistent the residual is with Gaussian noise.

We compute an odds ratio $Z$ to quantify the likelihood of a signal relative to the null hypothesis:
\begin{equation} \label{eq:def:Z:Modified}
Z(\{d\}|{\cal H}_1) \equiv \frac{p(\{d\}|{\cal H}_1)}{p(\{d\}|{\cal H}_0)} 
  = \int d\vec{\mu}\ p(\vec{\mu}|{\cal H}_1) \Like(\vec{\mu}|\{d\})\ .
\end{equation}
Here $p(\vec{\mu}|{\cal H}_1)$ is the prior probability for the parameters $\vec{\mu}$ and $\Like(\vec{\mu}|\{d\})$
is the \emph{likelihood ratio}:
\begin{equation} \label{eq:likelihood}
\Like(\vec{\mu}|\{d\}) = \prod_k \frac{\exp - \langle d - H(\vec{\mu}) | d - H(\vec{\mu}) \rangle_k / 2}
{\exp - \langle d | d \rangle_k / 2}\ .
\end{equation}

In addition to the odds ratio, we can also compute the posterior probability density 
(which we will abbreviate as simply the posterior)
for one or multiple parameters (marginalized over all other parameters). Let $x$ be one or more parameters in
$\vec{\mu}$ and $y$ be all other parameters, such that $\vec{\mu} = x \cup y$. Then, the posterior for $x$ is
\begin{equation} \label{eq:def:Posterior}
p({\cal H}_1|x) \equiv \frac{p(x|{\cal H}_1)}{Z(\{d\}|{\cal H}_1)} \ \int dy \;  p(y|{\cal H}_1) \Like(x,y)\ .
\end{equation}
%\editremark{[Do we like the $p_{\rm post}$ notation for posterior?]}

\subsection{Efficiently evaluating the likelihood}

We now show how the waveform decomposition derived above can be exploited
to speed up evaluations of the likelihood ratio.
In Eq.~(\ref{eq:decomposed_strain}) note that we have taken the observed strain in a detector $H_k(t)$
and rewritten it as a linear combination of harmonic mode time series $\hat{h}_{lm}(t)$.
The harmonic mode time series depend on the intrinsic parameters, but
the extrinsic parameters (apart from $t_k$) are entirely encoded in the coefficients of the linear combination.
Because computing the likelihood is just an inner product, which is a linear operator, 
we can pull these coefficients outside the inner product integral.
Thus, we need only compute inner products involving the $\hat{h}_{lm}$ and data $\{d\}$. 
If we store these, we can compute the likelihood for any extrinsic parameters by simply
recomputing the coefficients and reconstructing the linear combination.

To that end, we define the following quantities:
\begin{subequations}
%\label{eq:ComputeRhoViaInnerProductMatrix}
\label{eq:QUV}
\begin{align}
Q_{k,lm}(\itrprm,t_k) &\equiv \qmstateproduct{h_{lm}(\itrprm,t_k)}{d}_k \nonumber\\
&= 2 \int_{-\infty}^{\infty} \frac{df}{S_k(|f|)} e^{2\pi i f t_k} \tilde{h}_{lm}^*(\itrprm;f) \tilde{d}(f)\ , \\
{ U_{k,lm,l'm'}(\lambda)}& = \qmstateproduct{h_{lm}}{h_{l'm'}}_k\ , \\
V_{k,lm,l'm'}(\lambda)& = \qmstateproduct{h_{lm}^*}{h_{l'm'}}_k  \ .
\end{align}
\end{subequations}
Note that each $Q_{k,lm}(\itrprm,t_k)$ is computed for all $t_k$ with a single inverse Fourier transform, 
as in Eq.~(\ref{eq:IFFT}). Because GW detectors can localize a signal to a short time window 
on the order of milliseconds, The $Q_{k,lm}(\itrprm,t_k)$ will be sharply peaked as functions of $t_k$ 
and we need only retain the values for a narrow range of $t_k$. To allow for detector arrival times that differ
from the geocenter time, the range of $t_k$ for which we must store the $Q_{k,lm}$ is set by the light travel
time across Earth ($2R_{\oplus}/c\simeq 42 \unit{ms}$).   In this work we store the $Q_{k,lm}$ for a $300$ ms range of
$t_k$.  
Because time-shifting translates all $h_{lm}$ modes together, 
the $U_{k,lm,l'm'}(\lambda)$ and $V_{k,lm,l'm'}(\lambda)$ are independent of $t_k$
and are computed once by a straightforward use of Eq.~(\ref{eq:InnerProduct}).


By plugging Eq.~(\ref{eq:decomposed_strain}) into Eq.~(\ref{eq:likelihood}), taking a log and collecting terms, we obtain
\begin{widetext}
\begin{align}
\ln \Like(\itrprm; \etrprm) 
&= (D_{\rm ref}/D) \text{Re} \sum_k \sum_{lm}(F_k \Y{-2}_{lm})^* Q_{k,lm}(\itrprm,t_k)\nonumber \\
&   -\frac{(D_{\rm ref}/D)^2}{4}\sum_k
\left[
{
|F_k|^2 [\Y{-2}_{lm}]^*\Y{-2}_{l'm'} U_{k,lm,lm'}(\itrprm)
}
% \right. \nonumber \\ & \left.
 {
+  \text{Re} \left( F_k^2 \Y{-2}_{lm} \Y{-2}_{l'm'} V_{k,lm,l'm'} \right)
}
\right]
\label{eq:def:lnL:Decomposed}
\end{align}
\end{widetext}
Importantly, the intrinsic parameters $\itrprm$ enter only through the $Q_{k,lm}$, $U_{k,lm,l'm'}$ and $V_{k,lm,l'm'}$.
These are the dominant cost, as the require computing the orbital dynamics, the $h_{lm}$, inner product integrals
and inverse Fourier transforms. By contrast, the extrinsic parameters enter the $F_k$ and $\Y{-2}_{lm}$,
which are much simpler closed expressions.

Therefore, if we fix a point in the intrinsic parameter space we can compute the
$Q_{k,lm}(\itrprm,t_k)$, $U_{k,lm,l'm'}(\itrprm)$ and $V_{k,lm,l'm'}(\itrprm)$ only once,
vary the extrinsic parameters and compute the likelihood for only the cost of the $F_k$ and $\Y{-2}_{lm}$.
This allows us to efficiently integrate over the extrinsic parameters and obtain a marginalized posterior 
for the intrinsic parameters $p_{\rm post}(\itrprm)$ as in Eq.~(\ref{eq:def:Posterior}).
If we do this for a collection of points in the intrinsic parameter space, we can integrate over $\itrprm$ as
well and obtain the odds ratio $Z$ as in Eq.~(\ref{eq:def:Z:Modified}).
Note that the computation for each point in the intrinsic space is completely independent of the others.
This makes the algorithm embarrassingly parallel and given enough CPU cores the entire analysis can be
run in the time it takes to integrate over the extrinsic parameters (modulo some brief startup and post-processing steps).

The remainder of this section provides more details on the various steps of our algorithm.

%\subsection{Efficiently evaluating the likelihood}

% POINT: Strain formula
%The complex gravitational wave strain $h=h_+-ih_\times $ at any sufficiently distant point ($t,\vec{x}$) from a nonprecessing
 %binary can be efficiently represented using spin-weighted spherical harmonics $h_{lm}$ as
%\begin{align}
%\label{eq:h:Expansion}
%h(t-\vec{x}\cdot \hat{k}) = \sum_{lm} e^{-2i\psi} h_{lm}(t-\vec{x}\cdot \hat{k})  \Y{-2}_{lm}(\iota,-\phi_c)  \frac{d_{\rm ref}}{d}
%\end{align}
%where $d_{\rm ref}$ is some reference distance at which the $h_{lm}$ are evaluated.  
%In this expression, the vector $\hat{k}$ is the propagation direction ($\hat{k}=-\hat{n}$); the angles $(\iota,\psi)$ are the polar angles of the (fixed) orbital angular momentum direction
%relative to the propagation direction $\hat{k}$; $d$ is the distance to the source; and $\phi_c$ is the coalescence
%(orbital) phase. 
%Expressions for $h_{lm}$ are available in the literature 
%\cite{gwastro-pn-MultipoleMomentsNonspinning, gw-astro-mergers-approximations-SpinningPNHigherHarmonics}.  While many
%phenomenological approximations do not explicitly provide a harmonic representation, the leading order term ($h_{22}$) can be
%easily  identified from published literature by comparing those expressions to the above, evaluated at $\iota=\psi=\phi_c=0$.
  
%
%In terms of this complex strain, the response $H_k$ of the $k$th   gravitational wave detector to low-frequency
%radiation \cite{gwastro-GroundBasedResponse-Whelan2008} can
%be described by a linear combination of $h(t)$ and $h^*(t)$, weighted by time-sky-location-dependent coefficients $F_{+,\times}(t,\hat{n})$:
%\begin{align}
%H_k(t) &=F_{+,k}(t,-\hat{k}) h_+(t-\vec{x}_k(t)\cdot \hat{k}) \nonumber \\
% & + F_{\times,k}(t,-\hat{k}) h_\times(t-\vec{x}_k(t)\cdot \hat{k}) \nonumber \\
% &=  \frac{F_k(-\hat{k}) h(t-\vec{x}_k(t)\cdot \hat{k}) }{2} + \frac{F_k^*(-\hat{k})h^*(t-\vec{x}_k(t)\cdot \hat{k})}{2}
%\label{eq:Hh}
%\end{align}
%where in the second expression we adopt a complex antenna pattern $F_k=F_{+,k}+i F_{\times,k}$.  
%
%For simplicity, in this work we will treat the position $\vec{x}_k$ and antenna pattern $F_k$ of each instrument as
%constants.\footnote{The quality of this approximation is in direct proportion to the ratio $|\Delta x|/\lambda$ where
 % $\Delta x$ is the change in detector position over the waveform's duration and $\lambda$ is the longest significant
 % wavelength.   As either this ratio is a small quantity or the detectors are nearly insensitive to $\lambda$,
  %rotation often need not be included.  Moreover, when radiation \emph{does} need to be included, we anticipate a straightforward perturbative and stationary phase
  %approximation can augment the simple procedure described above, allowing the likelihood to be well approximated with
  %a factor few additional filters and scalars.
%}


%Substituting this expansion for $h$ [Eq. (\ref{eq:h:Expansion})] and the individual-detector response
%[Eq. (\ref{eq:Hh})] into the
%likelihood [Eq. (\ref{eq:def:LikelihoodRatio})], we find the likelihood can be expressed as
%\begin{widetext}
%\begin{align}
%\ln \Like(\lambda; \theta) 
%&= (d_{\rm ref}/d) \text{Re} \sum_k \sum_{lm}(F_k(-\hat{k}) e^{-2\psi} \Y{-2}_{lm}(\hat{n}))^* Q_{k,lm}(t-\hat{k}\cdot
%x_k)
%\nonumber \\
%&   -\frac{(d_{\rm ref}/d)^2}{2}\sum_k
%\left[
%{
 %\frac{1}{2}|F_k(-\hat{k})|^2 U_{k,lm,lm'}(\lambda)[\Y{-2}_{lm}(\hat{n})]^*\Y{-2}_{l'm'}(\hat{n})
%}
% \right. \nonumber \\ & \left.
 %{
%+
 %\frac{1}{2} \text{Re} V_{k,lm,l'm'} e^{-4i\psi}F_k^2 \Y{-2}_{lm}(\hat{n})\Y{-2}_{l'm'}(\hat{n})
%}
%\right]
%\label{eq:def:lnL:Decomposed}
%\end{align}
%\end{widetext}
%In this expression, we use $\hat{n}$ as shorthand for $(\iota,-\phi_c)$; define the constant, detector-dependent
%matrices $U,V$ by 
%\begin{subequations}
%\label{eq:ComputeRhoViaInnerProductMatrix}
%\begin{align}
%{ U_{k,lm,l'm'}(\lambda)}& = \qmstateproduct{h_{lm}}{h_{l'm'}}_k \\
%V_{k,lm,l'm'}(\lambda)& = \qmstateproduct{h_{lm}^*}{h_{l'm'}}_k  \;
%\end{align}
%\end{subequations}
%and define the \emph{filtered data series} $Q_{k,lm}(t)$ via 
%\begin{align}
%Q_{k,lm}(\tau) &\equiv \qmstateproduct{{\cal T}_{\tau} h_{lm}}{\hat{H}_k}_k \\
%&= 2 \int_{-\infty}^{\infty} \frac{df}{S_k(|f|)} e^{+2\pi i f \tau} \tilde{h}_{lm}(f)^* \tilde{\hat{H}}_k(f) \\
%\end{align}
%i.e., as the inner product of the time shifted $h_{lm}$ harmonic [$h_{lm}(t+\tau)\equiv [{\cal T}_\tau h_{lm}](t)$] with the data in the $k$th instrument.


% POINT: This is precomputable
%Critically, all intrinsic-parameter dependent terms in   Eq. (\ref{eq:def:lnL:Decomposed}) can be evaluated once for
%each $\lambda$.  Having calculated these constants ($U_{k,lm,l'm'},V_{k,lm,l'm'}$) and time series ($Q_{k,lm}(\tau)$),
%the likelihood can be efficiently evaluated for any choice of extrinsic parameters $\theta$.
%

% POINT: Computational requirements
%This procedure offers striking reductions in the total number of floating point operations needed to evaluate the
%likelihood.
%Given how precisely gravitational wave searches time-localize candidate events, only a \emph{short} stretch of data
%surrounding a candidate gravitational wave event will ever be examined in  detail.  Hence, rather than evaluate $\Like$ by operating on long-duration, densely sampled
%time- or frequency- series, our expression need only act on constants and a few short arrays.  After minimal preprocessing, the total number of
%floating point evaluations needed for any \emph{subsequent} likelihood evaluation at those same parameters $\lambda$ is
%dramatically reduced.  

%
%Since our method employs no assumptions about the structure of $h_{lm}$, this technique for organizing the likelihood
%calculation can be applied to any existing waveform model, including the effective-one-body approximants
%\cite{gw-astro-EOBspin-Tarrachini2012,gw-astro-EOBNR-Calibrated-2009}.  
%Unlike reduced-order or interpolation-based methods, no further coding, tuning, or calibration is required.  
%Our method requires no  additional model- or detector-dependent development  to achieve this performance improvement.  


\subsection{Placement over intrinsic parameters}
\label{sec:itr_placement}

\begin{figure}
\includegraphics[width=\columnwidth]{../Figures/linear_ellipse_placement.png}
\caption{\label{fig:linear_ellipse} \textbf{Intrinsic parameter placement:} We use an effective Fisher matrix to compute
an approximate ellipsoidal region of overlap $\geq 90\%$ with the masses reported by a detection pipeline.
We then fill this ellipsoid with discrete points and cut any with unphysical values of symmetric mass ratio $\eta$.
At each physical grid point, we marginalize the likelihood over all extrinsic parameters as described in
Sec.~\ref{subsec:extrinsic}.}
\end{figure}

In the current work, we restrict ourselves to considering non-spinning binaries in which we also neglect tidal
effects. Therefore, we need only two mass parameters to describe the intrinsic parameter space, for which we use
the symmetric mass ratio $\eta = m_1 m_2 / M^2$ 
and the chirp mass ${\cal M}_c = M \eta^{3/5}$ (where $M = m_1+m_2$ is the total mass).
Since detection searches will report masses for the candidate event, we use these
to guide which region of the intrinsic parameter space to explore.

Let $\left( {\cal M}^*,\eta^* \right)$ be the masses reported by a detection pipeline. We then perform an effective 
Fisher matrix calculation as described in~\cite{gwastro-mergers-HeeSuk-FisherMatrixWithAmplitudeCorrections,
gwastro-mergers-HeeSuk-CompareToPE-Aligned}
centered about this point.
This involves evaluating the overlap between our waveform with masses $\left( {\cal M}^*,\eta^* \right)$
and $\sim$ tens of nearby waveforms with different intrinsic parameters (while extrinsic parameters are held constant).
The measured overlap values are then fit with a multi-dimensional quadratic. 
The coefficients of this quadratic fit are called the effective Fisher matrix.
Like the standard Fisher matrix, the effective Fisher matrix serves as a quick, crude estimate of expected parameter
estimation performance and can be used to predict surfaces of constant overlap, which will in general be ellipsoids.
In this work, we use the effective Fisher matrix to approximate the region of intrinsic parameter space
which will have overlap $\geq 90\%$ with the masses $\left( {\cal M}^*,\eta^* \right)$.

Once we have defined this $90\%$ overlap ellipsoid, we must fill it with a set of discrete points 
at which we will compute the likelihood. First, we specify the total number of intrinsic parameter points we wish to
place, here  200 points. Then, these points are arranged within a unit sphere. In this work, we placed
points along 20 radial ``spokes'', with 10 points per spoke. Along each spoke, the points are placed 
uniformly in radial distance. Now, the eigenvalues and eigenvectors of the effective Fisher matrix tell us the lengths
and orientations of the axes of the $90\%$ overlap ellipsoid. We use these to deform and rotate our set
of points in the sphere to a set of points in the $90\%$ overlap ellipsoid.
%% One subtlety is that the only physically-meaningful values of $\eta$ are in the range $(0, 0.25]$, but the effective Fisher
%% approach does not account for this physical cutoff. Therefore, for near equal mass binaries where $\eta^* \simeq 0.25$
%% part of the $90\%$ overlap ellipsoid may have unphysical $\eta$ values. 
Once we have filled our ellipsoid, we remove
any points that have unphysical $\eta$ ($>1/4$).   %To counteract this somewhat, 
We ensure that we always place spokes in our ellipsoid along the direction of constant $\eta$,
so that we always have many points along this boundary of the parameter space.
Fig.~\ref{fig:linear_ellipse} illustrates this placement of intrinsic parameter points.

Note that this intrinsic placement routine could be modified in many different ways. 
%% For example, 
%% we could place points uniformly in volume, rather than uniformly in radius. This would place more points
%% towards the edge of the ellipsoid, while we chose uniform-in-radius to get more points near the center.
%% We could also place points randomly inside the ellipse (using uniform-in-volume, uniform-in-radius, 
%% or any other distribution). We chose the spoked placement so that we can always ensure near-equal mass binaries
%% will have many points near the $\eta = 0.25$ boundary of the parameter space.
%% One might also use larger or smaller ellipsoids, use multiple ellipsoids centered at different points if there are concerns
%% about bias in the masses reported by the detection pipeline, choose points according to a metric, 
%% or consider any number of other refinements. 
We plan to refine the intrinsic parameter placement in future work.





\subsection{Integrating over extrinsic parameters}
\label{subsec:extrinsic}

% POINT: General integral form
Because precomputed quantities allow us to efficiently evaluate the likelihood $\Like(\lambda,\theta)$ as a function of
$\theta$, we first integrate the likelihood $\Like(\lambda,\theta)$ over all extrinsic parameters $\theta$:
\begin{eqnarray}
\LikeRed(\lambda) = \int \Like(\lambda,\theta) p(\theta) d\theta
\end{eqnarray}
where $p(\theta)$ is our prior over the extrinsic  parameters.  We assume the sources analyzed are randomly-oriented and
randomly distributed in the universe out to $D_{\rm max}=300 \unit{Mpc}$.  
%% \begin{eqnarray}
%% p(d)=  3 d^2/d_{\rm max}^3 \quad d_{\rm max} = 300\unit{Mpc} \\
%% \end{eqnarray}
% POINT: Specific separable prior, sampling priors approach
To be concrete, we evaluate the reduced likelihood $\LikeRed$ by integrating over sky position $\Omega_{\rm sky}$ represented
as right ascension $\alpha$ and declination $\delta$; angular momentum orientation, measured by inclination $\iota$ and
polarization angle $\psi$;  and coalescence phase $\phi_{\rm c}$, using the separable prior implicitly provided by the
following integral:
\begin{eqnarray}
\LikeRed(\lambda) = \int \frac{dt}{T_{\rm window}} \frac{d^2 dd d\Omega_{sky} }{V_{\rm max}} \frac{d \cos \iota d\phi_c}{4\pi} \frac{d\psi}{\pi} \Like(\lambda,\theta)
\end{eqnarray}
%
In this expression and our calculations, we adopt a maximum distance $d_{\rm max}$ (here $300 \unit{Mpc}$) and a time window
$ T_{\rm window}$  (here, $300\unit{ms}$)  surrounding the event.  

% POINT: MC
With the exception of time (described below), we evaluate these integrals and reconstruct the posterior distribution using Monte Carlo integration
\cite{book-mm-NumericalRecipies,peter1978new}.   To establish notation used below, we briefly review the general principles  underlying Monte Carlo
integration.  If $p_s$ is a distribution which is never zero when $p>0$, then 
\begin{eqnarray}
\LikeRed(\lambda) = \int \frac{\Like(\lambda,\theta) p(\theta)}{p_s(\theta)} [p_s(\theta) d\theta]
\end{eqnarray}
% POINT: Integral value and its error
% 
If we draw $N$ random samples $\theta_q$ from $p_s$, we can estimate the value $\hat{{\cal L}}_{\rm red}$ of $\LikeRed$ and its error using the
expectation value and central limit theorems for independent, identically-distributed random variables:
\begin{eqnarray}
w_q \equiv \frac{\Like(\lambda,\theta_q) p(\theta_q)}{p_s(\theta_q)} \\
\hat{{\cal L}}_{\rm red}(\lambda) \equiv \frac{1}{N} \sum_q w_q = \left<w\right> \\
\sigma_{\LikeRed}^2 = \left<w^2\right> - \left<w\right>^2
\end{eqnarray}
% POINT: Recombinable
Being a pure Monte Carlo integral, we can  combine the results of multiple independent draws of $N$ events, even if these
evaluations adopted different sampling prior.  As a result, this approach is highly parallelizable.  
% Combining results from multiple runs: because pure Monte Carlo, can combine results after multiple runs, using formula


% POINT: PDF reconstruction
The weighted samples also provide an estimate of the marginalized one-dimensional cumulative distributions $P(<x)$ at
fixed $\lambda$, where $x$ is one of the extrinsic variables in $\theta$:
\begin{eqnarray}
\hat{P}(<x) \equiv \frac{1}{\sum_q w_q} \sum_q w_q \theta(x-x_q)
\end{eqnarray}
[This formula follows by performing Monte Carlo integration on the parameter volume $<x$, keeping track of overall
  normalization.]  
In the limit of many samples, this discontinuous estimate should converge to a smooth, marginalized posterior
  distribution.  
%
For any set of samples and any $x$, the error in $\hat{P}$ follows from the (correlated) statistics of the Monte Carlo
integrals in its numerator and denominator; see the Appendix for a rigorous discussion. 
% POINT: neff
In the typical case that all samples $x_q$ are distinct, the unique sample with the largest weight corresponds to the
largest discontinuity in $\hat{P}$.  The magnitude of this discontinuity, or equivalently its inverse $n_{\rm eff}$,
provides a practical measure of how reliable we expect this one-dimensional posterior to be:
\begin{eqnarray}
n_{\rm eff} \equiv \frac{\sum_q w_q}{\text{max} w_{\rm q}}
\end{eqnarray}
Equivalently, the ``effective number of samples'' $n_{\rm eff}$ measures how many independent samples produce similar
weights near the largest observed weight.  
%
Motivated by the effective number of samples and by analogy to the error in a one-dimensional unweighted cumulative
distribution, we expect the standard deviation in our estimate $\hat{P}$ for $P$ to be comparable to or less than the
standard deviation of an \emph{equally-weighted} set of $n_{\rm eff}$ samples,  derived from the binomial distribution:
\begin{eqnarray}
\label{eq:ErrorEstimateCumulative}
\left<\hat{P}^2\right > - \left<\hat{P} \right>^2 \lesssim \frac{P(1-P)}{n_{\rm eff}}
\end{eqnarray}
We \editremark{will} validate this simple estimate.  


%
Unless otherwise indicated, we draw samples using a separable sampling prior $p_s(\alpha) =
\prod_{\alpha}p_{s,\alpha}(\theta_\alpha)$, with each factor $p_{s,\alpha}$ equal to the corresponding prior in that
dimension. 
%% {color{blue} These aren't exceptions: The sampling prior is separable in all parameters unless using a
%% skymap, and then it's effectively $\alpha,\delta\rightarrow\tau$ where $\tau$ is a time delay... 
For some parameters, we adopt a sampling prior that is different than the physical prior, guided by a priori physical
choices; search input;  or past Monte Carlo experience.  These parameters include 
distance (either uniformly or adaptively sampled) and sky position (either uniformly, adaptively, or via a skymap), as
described below.
%

% POINT: How many runs?
Currently, we perform $n_{\rm trials}$ ($=10$) evaluations at each mass point, terminating when either $N$ iterations
crosses a threshold ($=10^6$) or to some fixed
$n_{\rm eff}$ threshold ($=1000$), whichever comes first.  
%
To allow for intrinsic differences in sampling prior between evaluations, the $n_{\rm trials}$ evaluations at each mass
point are
recombined weighted by their (estimated) error: if $\LikeRed_k$  and $\sigma_{\LikeRed,k}$ are the reported integral
values and error estimates for $k=1\ldots n_{\rm trials}$, then
\begin{eqnarray}
\hat{\LikeRed}(\lambda) = \frac{\sum_k \LikeRed_k/\sigma_{\LikeRed,k}^2}{\sum_k 1/\sigma_{\LikeRed,k}^2}
\end{eqnarray}


% POINT: Storage
To mitigate storage requirements, we discard  low-weight samples before recording results.  Low-weight samples are identified by sorting  $w_1<w_2<\ldots$;
constructing the cumulative $W_=\sum_{q\le k} w_q$; and discarding all samples with $W_k/W_N<10^{-4}$.  

\subsubsection{Adaptive Monte Carlo integration}

To better concentrate our samples in regions of high significance, we implemented a simple adaptive Monte Carlo procedure, adjusting
the sampling prior  based on measured weights $w_k$.   The adaptive sampler was used only for selected parameters (i.e.,
distance and sky location), not ubiquitously.   In the long term, we expect to apply more sophisticated adaptive
algorithms, as described in the literature \cite{book-mm-NumericalRecipies,peter1978new}.  For reference, we describe
the specific adaptive algorithm implemented used in this work.


% POINT: Adaptive procedure
Every $n_{\rm chunk}$ samples, we reconstruct the one-dimensional sampling priors in each adapting dimension, using
the last $n_{\rm chunk}\times n_{\rm h}$ samples.    In each dimension ($x$), we subdivide the full range
into $n_{\rm bins}$ equal-length bins $X_\alpha$, then evaluate a tempered, weighted histogram
\begin{eqnarray}
W_\alpha = \frac{\sum_{q} w_q^\beta \Theta_{X_\alpha}(x_q)}{\sum_q w_q^\beta}
\end{eqnarray}
where $\Theta_Y(x)=1$ if  $x$ is in Y and zero otherwise,  and where $\beta$ is a  parameter to moderate the dynamic
range of $w$, described below.  
%
We then \emph{smooth} the array $W$, convolving it with a uniform distribution $n_{\rm smooth}$ bins across.
%
We generated an estimated discrete sampling prior by average the smoothed array $W$ with a uniform distribution with weight $s$:
\begin{eqnarray}
\hat{W}_\alpha = s W + (1-s)/n_{\rm bins}
\end{eqnarray}
% POINT: Discrete to continuous
Finally, we transformed from this discrete, bin-by-bin representation to a continuous integral by (a) constructing
a one-dimensional sampling prior $p_{s,x}(x)$  by interpolating $W^*/\Delta x$ between bin centers, then (b) constructing the
one-dimensional inverse $P_{s,x}(<x)^{-1}$ by integrating $P_{s,x}'(x)=p_{s,x}(x)$.  
%
The latter process ensures that the samplining prior and inverse CDF used to generate random samples are
self-consistent.   
%


% POINT: Specific choices used here
As configured for this paper, we used $n_{\rm bins}=100, n_{\rm hist}=50,n_{\rm smooth}=10,s=0.1$.  We chose the
tempering exponent using the network SNR $\rho$ reported by the search:
\begin{eqnarray}
\beta = \text{min} \left[ 0.8, 4 \frac{\ln (10 n_{\rm chunk})}{\rho^2} \right]
\end{eqnarray}
This choice attempts to mitigate the large dynamic range of $L$ to a scale comparable to the number of samples used in
each histogram.  
%\editremark{We should have changed 10 to nhist...}


%% ---- \editremark{Continue here}

%% * General concept

%% * Specific dimensions adapted

%% * tunings adopted: 100 bins, floor, tempering, history

%% * discard values after burn-in? Not yet....




\subsubsection{Using search results to target specific areas of the sky}

%
Before we begin, gravitational wave search pipelines have usually already identified candidate event times in two or
more interferometers' data, typically to much less than the light crossing time of the earth.  Triangulation limits the
set of candidate sky locations consistent with these results.  Rather than waste time sampling irrelevant sky positions,
we can use search input to immediately target our Monte Carlo at a region of high support.


As a concrete example, the \BS{} pipeline \cite{gw-astro-Bayestar} rapidly processes the results of a gravitational-wave search to identify candidate sky locations consistent with a gravitational wave event with minimal assumptions about the waveform parameters.
%
This code produces a posterior distribution $p_{BS}(\alpha,\delta)$ for the GW signal sky location in a discretized, interlocking, equal-area
grid of pixels corresponding to sky regions, with probabilities for each pixel (a healpix skymap). The refinement of the
grid is variable and scales with the resolution required to resolve the features of the posterior in $\alpha$ and
$\delta$.  
%
%
Using a  distribution, we can construct a sampling prior  $p_s(\alpha,\delta)$.
%
However,  the typical sky resolutions provided by \BS{} are significantly finer than the resolving power of current
gravitational-wave detector networks;   moreover, we are free to adopt higher resolutions as needed.  Hence,  for
simplicity we adopt a purely discrete sky, instead of sampling continuously: the samples are selected from the set of sky
region centers as calculated by the healpix library. These regions are sampled appropriately by their weight in the
posterior.   This method can in principle break down when the true posterior is both large and rapidly changing across
sky pixels.  We will demonstrate efficacy and limitations of a discrete sky by example, below.
%  One possible disadvantage to this is for very sharp features in the posterior, it is possible that this method will fail to find the exact peak of the likelihood function in the sky coordinates. {\color{blue} Can we or should we quantify this?} 

\ForInternalReference{
{\color{blue} The remainder of this explanation is technical and may not be appropriate for the paper. It is here for posterity.} Specifically, the \BS{} pipeline provides a set of $N$ pixels represented as a normalized array of probability mass $s_k$, each index $k$ associated with a pixel centered on $(\alpha_k,\delta_k)$.
%
% Healpy review: http://planck.oat.ts.astro.it/planck/software_and_manuals/Healpix_1.22/Healpix_intro.pdf
%   npix = 12*nside^2 = 49152     %% 4 Pi/(49152)/(Degree)^2 // N
%   nside = 64 (for us)
To generate sky position samples, we generate instead a set of indices corresponding with each index having a multiplicity of $M_k=\frac{s_k}{s_{\text{min}}}\simeq p_s(\alpha_k, \delta_k)$. This represents the probability mass of the region up to an overall quantization error relative to the smallest $s_k$ chosen ($s_{\text{min}}$). The minimum pixel probability is chosen small enough to well represent the quantized probability distribution. {\color{blue} ...and we have a plot of this if necessary}. Finally, a random element of the set is selected and the index chosen is converted back to its sky coordinates $(\alpha_k,\delta_k)$. 
% The point is to avoid rejection sampling which is slow
}

%% POINT: Very high SNR: finer skymaps are available, and we can use

\subsubsection{Time marginalization}
\label{sec:time_marg}

Having already computed all ingredients needed (i.e., $Q_{k,lm}(t)$), we can evaluate the likelihood versus time $\Like(t)$ cheaply, by array
addition operations.  Hence, rather than performing Monte Carlo integration, we can likewise
efficiently \emph{integrate} over time.  

Specifically,  for every sky location drawn by the adaptive Monte Carlo routine, we compute the
corresponding time shift between geocenter and each detector with Eq.~(\ref{eq:t_k}).
To evaluate the value of $\ln {\cal L}$ for some geocenter time $t_{\rm geo}$,
we simply look up the corresponding value of the $Q_{k,lm}(\itrprm,t_k)$ from our precomputed values and plug them
into Eq.~(\ref{eq:def:lnL:Decomposed}).  
%
The resulting time series can be trivially integrated over  $t_{\rm geo}$ by performing a finite sum; see
Appendix \ref{ap:DiscreteData} for details. 





\subsection{Postprocessing: intrinsic priors, interpolation, and the final results}

% POINT: Interpolating in intrinsic parameters
At this point, we have evaluated $\LikeRed(\lambda)$ over a structured grid of intrinsic parameters $\lambda_\mu$.  
%
By construction,  the values $\LikeRed(\lambda)$ have small statistical errors (e.g., less than $1\%$).   Using standard
interpolation packages, we interpolate $\LikeRed(\lambda)$ throughout the sampled grid.\footnote{For example, for linear
spoked interpolation in two dimensions, we adopt polar coordinates that are compatible with the 2d polar grid.}
%
Combined with the prior $p(\lambda)$ over intrinsic parameters, we evaluate the overall evidence $Z$ and posterior
distribution $p_{\rm post}(\lambda)$ over intrinsic parameters via
\begin{eqnarray}
Z = \int d\lambda p(\lambda) \LikeRed(\lambda) \\
p_{\rm post}(\lambda) = \frac{1}{Z} p(\lambda) \LikeRed(\lambda)
\end{eqnarray}
%

% POINT: What about posteriors in 
Similarly, we can construct a posterior distribution over some \emph{extrinsic} parameter $x$ by averaging the one-dimensional posterior estimates derived at
each mass point, weighting each by the prior $p(\lambda)$:
\begin{eqnarray}
P_{\rm post}(<x) = \frac{\int d\lambda P_\lambda(<x) \LikeRed(\lambda) p(\lambda)}{\int d \lambda p(\lambda) \LikeRed(\lambda)}
\end{eqnarray}
This weighted average requires interpolating both $\LikeRed(\lambda)$ and $P_\lambda(<x)$ over all $\lambda$.  


% POINT
%\editremark{Lazy, current approximation}
To a good approximation, however, the intrinsic and extrinsic parameter distributions often separate after marginalizing
in time.  In other words, after marginalizing in time, the extrinsic parameter distributions are nearly independent of $\lambda$.  
In that case, the level of caution exercised above is unwarranted:  each individual
extrinsic parameter distribution provides a reliable estimate of the posterior.  As a crude approximation to the posterior distribution $p_{\rm post}(\theta)$, we  simply \emph{combine} all
weighted samples $(x_{\mu,q},w_{\mu,q})$ from all mass points $\mu$:
\begin{eqnarray}
\hat{P}(<x) = \frac{\sum_{q\mu} w_{q,\mu}\theta(x-x_{\mu,q})}{\sum_{q\mu} w_{q\mu}}
\end{eqnarray}
[This expression agrees with the general approach, for the specific prior implied by the uniform, spoked grid.]
{\color{blue} Feedback?}




\section{Results: Production environment}
\label{sec:Results}

We present a proof-of-principle study of our parameter estimation pipeline, employed in a environment which should
resemble the steps taken after the identification of search triggers. To this purpose, a mock data challenge was
constructed to exercise both the low latency gravitational wave searches as well as the parameter estimation follow-up
processes expected to be applied to GW candidates identified by the search \cite{first2years}. This challenge proceeded
in several stages, each desgined to emulate the anticipated workflow in a low latency BNS detection environment. 

Before beginning, data was constructed consistent with the expected  (median) sensitivity for  2015
from \cite{LIGO-Inspiral-Rates,LIGO-2013-WhitePaper-CoordinatedEMObserving}, documents that describe the evolution of
sensitivity of second-generation detectors.   A population of GW signals from binary neutron stars was added into the
data, as described in \cite{first2years} and reviewed here.   Events in this set (the 2015 MDC data) were distributed
isotropically on the sky, and in uniformly in volume out to \textbf{219} \unit{Mpc}. The BNS injections had uniform
random component masses in $1.2 M_\odot-1.6 M_\odot$ and randomly oriented spins with the dimensionless magnitude not
exceeding 0.05. 
These  (precessing) binary signals  were generated via precessing
SpinTaylorT4 templates at 3.5 PN order \cite{BCV:PTF}, including all known post-Newtonian modes \cite{gw-astro-mergers-approximations-SpinningPNHigherHarmonics}.  


First, the \gstlal{} BNS search was performed over the MDC set, identifying events for further follow up.  We present
parameter estimation results from 450 events seleted randomly from a set of recovered events with false alarm rate
smaller than one per century. This threshold is motivated by the selection criteria outlined
in \cite{LIGO-2013-WhitePaper-CoordinatedEMObserving}. From the reported  search pipeline results, we use the $\mc$ and
$\eta$ coordinates from which the intrinsic search space is extrapolated, a the window for time marginalization (see
Section \ref{sec:time_marg}) is formed around the reported coalescence time $t_r$.  The search pipeline also provides reference power spectral densities $S_n(f)$ utilized in evaluating the likelihood.

In addition to the information provided by the search pipeline, we also make use of the posterior probability of $\alpha$ and $\delta$, provided by \BS. It is expected that such information will be available within a few minutes of trigger identification. The gains in time to convergence in the MC integral are expected to outweigh the loss of time waiting for \BS{} to finish processing if the \BS{} posterior is used as a sampling function for the sky location.

%
% POINT: What code was used
Unless otherwise stated, the likelihoods were evaluated using nonspinning TaylorT4 templates at 3.5 post-Newtonian
order, including only the $(2,\pm 2)$ and $(2,0)$ modes.    As we will discuss at length below, this signal model does \emph{not} include all degrees of
freedom permitted to the binary in the data.  
When evaluating the likelihood,   waveforms started at $f_{\rm low}=40\unit{Hz}$.  The inner product (equation \ref{eq:InnerProduct}) integration uses an inverse power spectrum filter targeting the frequency range $[f_{\rm min},f_{\rm max}]=[f_{\rm min}, 2000\unit{Hz}]$, constructed from the measured power spectrum as described in Appendix \ref{ap:DiscreteData}.  

For each event identified by the GW search pipeline, we distributed the intrinsic points according to the procedure described in Section \ref{sec:itr_placement} and evaluated the Monte Carlo integral $10$ times for each mass point. We outline the form of the priors and sampling functions for each parameter in Table \ref{tbl:priors}.

\begin{table}
\begin{tabular}{r|c}
\hline \\
parameter & physical prior \\
\hline \\
distance ($D$) & uniform in volume \\
inclination ($\iota$) & uniform in $cos\iota$, $\iota$ between $(-1, 1)$ \\
polarization ($\psi$) & uniform in $(0, 2\pi)$ \\
orbital phase ($\phi$) & uniform in $(0, 2\pi)$ \\
declination ($\delta$) & uniform in $\cos\delta$, $\delta$ between $(-\pi, \pi)$ \\
right ascension ($\phi$) & uniform in $(0, 2\pi)$ \\
\hline
\end{tabular}
\caption{\label{tbl:priors}Prior distributions used for the extrinsic parameters in the 2015 MDC study.}
\end{table}

Only distance used an adaptive sampling function, starting initially with a constant function. Our distance prior was uniform in volume out to $d=300\unit{Mpc}$. The declination and right ascension were sampled from the posterior provided by \BS. As used here, the skymap had \textbf{$12\times 64^2$ pixels, roughly one per square degree.}


%\subsection{Advanced Detector Mock Data Challenge}
%\label{sec:BNS_2015_MDC}

%The evolution of the sensitivity of the second generation detectors has been described in \cite{LIGO-Inspiral-Rates,LIGO-2013-WhitePaper-CoordinatedEMObserving}. A mock data challenge using data which posseses a sensitivity which matches the median curve in 2015 from that document was used to test our pipeline. In addition to this data set, a population of GW signals from binary neutron stars was added into the data. The population parameters are outlined here, as well as in \cite{first2years}.

%The \gstlal{} pipeline was used to select events for further study. We present parameter estimation results from 450 events seleted randomly from a set of recovered events with false alarm rate smaller than one per century. This threshold is motivated by the selection criteria outlined in \cite{LIGO-2013-WhitePaper-CoordinatedEMObserving}.


\subsection{Detailed investigation of one event}
%% WHAT FIGURES CURRENTLY ARE : 
%     - https://ldas-jobs.phys.uwm.edu/~evano/skymap_reruns/coinc_id_833/ 
%     : v2 = bayestar as prior and sampler, adapt in distance
%    
%% m1 = 1.21992194653 (Msun)
%% m2 = 1.20069205761 (Msun)
%% s1x = 0.00190996297169
%% s1y = 0.00721042277291
%% s1z = -0.00683548022062
%% s2x = -0.00129437400028
%% s2y = -7.87806784501e-05
%% s2z = 0.00192063604482
%% lambda1 = 0.0
%% lambda2 = 0.0
%% inclination = 2.62300610542
%% distance = 112.5338974 (Mpc)
%% reference orbital phase = 4.29952907562
%% time of coalescence = 966822123.762
%% detector is: H1
%% Sky position relative to geocenter is:
%% declination = 0.678935229778 (radians)
%% right ascension = 4.10562181473 (radians)
%% polarization angle = 6.19411420822

Despite providing complete results in less than one hour, our strategy provides extremely well-sampled distributions and
evidence, with small statistical error.   To illustrate its performance, we have selected a single event, whose
parameters are provided in Table \ref{tab:FiducialEvent:Parameters}.   
%



% POINT: Consistency 
The repeated independent evaluations naturally produced by our algorithm provide a simple self-consistency check,
allowing us to quantify convergence empirically.  
Specifically, at each of the 111 mass points automatically targeted for investigation by our algorithm for this event, we
independently evaluated the integral $\LikeRed$ [Figure \ref{fig:FiducialEvent:LikelihoodVersusMchirpEta}] and construct one- and two-dimensional posterior distributions, from
10 independent evaluations [e.g., Figure \ref{fig:FiducialEvent:Triplot:TriggerMasses}].    
%
As illustrated by example in Figure \ref{fig:FiducialEvent:Triplot:TriggerMasses}, these fixed-mass posterior
distributions are smooth; overlap the true parameters \editremark{Need to add injection cross}; and have errors
consistent  with the previously-reported estimate \editremark{prove me}.  
%
Moreover, as illustrated by Figure \ref{fig:FiducialEvent:Cumulatives:Comparison:TriggerMasses}, each of the 10
independent evaluations make posterior predictions that are consistent with one another.   
%
Finally, as illustrated by Figure \ref{fig:FiducialEvent:Integral:ErrorEstimate}, for each mass point the 10 values of $L_{\rm red}$
are consistent with one another to roughly $1\%$.  
%
Keeping in mind our final reported results combine both all mass points and all 10 evaluations at each mass point, we
anticipate relatively little uncertainty due to sampling error in our current configuration.  


\begin{table}
\begin{tabular}{l|ll}
Parameter & True & Search \\ \hline
$m_1 (M_\odot)$ &  1.22 & 1.26 \\
$m_2 (M_\odot)$ &  1.20 & 1.16 \\
$|\chi_1| $ & 0.01  & 0 \\
$|\chi_2| $ & 0.002 & 0 \\
$d (\unit{Mpc}) $ & 112.5 & - \\
$\iota $ & 2.62 & - \\
($\alpha,\delta$) & (4.106,0.6789) &\\ 
$\rho_{\rm search}$ & 10.85 \\
\end{tabular}
\caption{\label{tab:FiducialEvent:Parameters}\textbf{Fiducial event: True and trigger parameters}: The physical parameters of our injected event, compared
  with the parameters provided by the search and used to target our parameter estimation followup.
}
\end{table}

% POINT: Posterior in mc, eta
To construct confidence intervals in $\mc,\eta$, we adopt a uniform prior in $m_1,m_2$ with $m_1,m_2\in[1,30]M_\odot$
and $m_1+m_2\le 30$: inside the specified region, the prior density is $p(m_1,m_2)dm_1dm_2=2 dm_1 dm_2/(28 M_\odot)^2$.
Changing coordinates using the Jacobian $d(m_1,m_2)/d(\mc,\eta)= \delta \mc/M^2 = \delta \eta^{6/5}\mc^{-1}$ where
$\delta = (m_1-m_2)/M = \sqrt{1-4\eta}$, we find the prior density in $\mc,\eta$ coordinates is
\begin{eqnarray}
p(\mc,\eta) d\mc d\eta =  \frac{1}{392 M_\odot^2} \frac{\mc}{\eta^{6/5}\sqrt{1-4\eta}}
\end{eqnarray}
%
In other words, due to a coordinate singularity, the prior in $\eta$ diverges near the equal-mass line.  
%
This coordinate singularity has a disproportionate impact on comparable-mass binary posteriors.


% POINT
The top panel of Figure \ref{fig:FiducialEvent:LikelihoodVersusMchirpEta} shows contours of the reduced likelihood $\LikeRed$ versus
$\mc,\eta$, derived by interpolating between our discrete grid.   As expected, these contours largely agree with the
Fisher matrix used to construct our grid. 
%
For comparison, the bottom panel of Figure \ref{fig:FiducialEvent:LikelihoodVersusMchirpEta} shows the contours of
$p(\mc,\eta)\LikeRed(\mc,\eta)$, including the strong coordinate singularity near the equal-mass line.  
%
Despite the large log-likelihood and the good agreement between the Fisher matrix and likelihood contours, this
coordinate singularity leads to significant differences between a naive 
Fisher-matrix estimate and the posterior distribution.



\begin{figure*}
\includegraphics[width=\textwidth]{../Figures/v2runs_coinc_id_833_ILE_triplot_MASS_SET_0}
\caption{\label{fig:FiducialEvent:Triplot:TriggerMasses}\textbf{Posterior distribution in intrinsic parameters, assuming known masses}: For our fiducial event, our predicted
  distribution of extrinsic parameters $d,RA=\alpha,DEC=\delta,\iota,t,\phi,\psi$, for clarity evaluated assuming at the
  mass parameters identified by the search.  Extremely similar distributions are recovered at each mass point. 
\ForInternalReference{  \emph{Suggest: show d-cos iota, skymap, and phi-psi only, not full triplot}}
}
\end{figure*}


\begin{figure}
%\includegraphics[width=\columnwidth]{../Figures/v2runs_coinc_id_833_mchirp_eta_logevidence}
\includegraphics[width=\columnwidth]{../Figures/fig-mma-manual-coinc833-LReducedVersusMcEta}
\includegraphics[width=\columnwidth]{../Figures/fig-mma-manual-coinc833-PosteriorMcEta}
\caption{\label{fig:FiducialEvent:LikelihoodVersusMchirpEta}\textbf{Marginalized likelihood and posterior distribution
    versus component masses}: Illustration of intrinsic parameter estimation of a spinning NS-NS event using
    nonspinning templates.  \emph{Top panel}: 
For our fiducial event,
  contours of the log of integrated likelihood $\ln \LikeRed$ versus component masses, represented in $\mc,\eta$
  coordinates.  Points indicate the mass grid adopted; thin contours show isocontours of the (interpolated)
  $\ln \LikeRed=35,36,37,38,39$; and the solid green point shows the injected parametes.   For comparison, the thick black curve corresponds to the 90\% confidence interval
  predicted by combining the network SNR reported by the search ($\rho_{\rm search}$ in Table \ref{tab:FiducialEvent:Parameters}) and the (effective)
  Fisher matrix used when placing test points, as in \cite{gwastro-mergers-HeeSuk-FisherMatrixWithAmplitudeCorrections,gwastro-mergers-HeeSuk-CompareToPE-Aligned,gwastro-mergers-HeeSuk-CompareToPE-Precessing}.  
\emph{Bottom panel}: The 90\% (blue) and 68\% confidence interval (black) derived from the posterior $\LikeRed(\mc,\eta)
p(\mc,\eta)/Z$.  For comparison, this figure also includes the same naive Fisher matrix estimate, shown as a thick black
 line.  
 \editremark{Once OK with Ben Farr, add curves from MCMC TaylorF2}
}
\end{figure}



\begin{figure}
\includegraphics[width=\columnwidth]{../Figures/v2runs_coinc_id_833_cumulative-multiplot-distance-MASS_SET_0}
\caption{\label{fig:FiducialEvent:Cumulatives:Comparison:TriggerMasses}\textbf{Sampling error analysis I: One-dimensional cumulative distribution at fixed mass}:  For each of the 10 independent
  instances used at  the mass  parameters   identified by the search, a plot of the one-dimensional cumulative
  distribution in distance.  These distributions agree to within a few \textbf{(quantify)} percent, qualitatively
  consistent with a naive estimate based on $1/\sqrt{n_{\rm eff}} \simeq 4\%$.   Combining all 10
  independent runs, we expect the final distance posterior has even smaller statistical sampling error
  (\textbf{quantify} $\simeq X/\sqrt{10}$).  Our final posterior
  distributions, having support from several mass points, should have smaller statistical error still.
 \editremark{Once OK with Ben Farr, add curves from MCMC TaylorF2}
}
\end{figure}

\begin{figure}
\includegraphics[width=\columnwidth]{../Figures/fig-mma-paper-10202-IntegralErrorVersusNeff}
%\includegraphics[width=\columnwidth]{../Figures/fig-mma-manual-v2_coinc_833-IntegralVarianceVersusNeff}
\caption{\label{fig:FiducialEvent:Integral:ErrorEstimate}\textbf{Sampling error analysis II: Integral error}: Blue
  points show the standard deviation in $L_{\rm red}$ versus $n_{\rm eff}$ reported for each of the individual code
  instances (1470 trials, corresponding to 147 mass points, 10 times each).  Red points shows the composite error
  $\sigma_{L}$ derived by weighting each of the 10 trials at each mass point, versus the sum of all $n_{\rm eff}$ in
  each trial.   \editremark{Currently showing 10202 - not clear why this iteration is so bad; previous skymap versions had
  sub-percent errors in the composite results? }
%% For each of the 111 mass points evaluated for the fiducial
%%   event, a scatterplot of the mean number of effective samples $n_{\rm eff}$ versus the standard deviation in $\ln
%%   L_{\rm red}$, where at each mass point the mean and standard deviation are calculated over the 10 independent
%%   evaluations performed.  Considering our final prediction for $L_{\rm red}$ combines all 10 events, this figure
%%   suggests $L_{\rm red}$ is known to better than $1\%$ for each mass point.  
% Additionally, this figure illustrates just how many effective samples are available for *each* mass point: thousands.
}
\end{figure}



\subsubsection{Controlled, zero-spin tests}
Though the source investigated here contained two (precessing) spins, our signal model explicitly omitted any spin
effects.  Recent calculations suggest advanced interferometers can weakly constrain astrophysically plausible spins in
binary neutron stars  \cite{2014PhRvL.112j1101F,2014arXiv1404.3180C}. 
%
For this reason, we have carefully repeated our calculations using a  nonspinning source with otherwise identical parameters, inserted into (a) no
noise and (b) the same data used above.   
%
For simplicity, rather than perform a search, we employed the same skymap and coincidence data identified
above.  \editremark{explain triggering issues: time offset needs to be fixed}
%
Results of this comparison are shown in Figure \editremark{XXX}



\begin{figure}
\caption{\textbf{Marginalized likelihood and posterior distribution versus component masses II: Zero spin}:
As Figure \ref{fig:FiducialEvent:LikelihoodVersusMchirpEta}, but for an injection without spin superimposed on (a) identical data
[black] and (b) zero noise [blue]. 
}
\end{figure}



\subsubsection{Validating technical assumptions}



* \editremark{DISCRETE SKY  - resolution limitations? NOT DISCRETE SKY?}

* Validating seperability of posteriors

\subsection{Ensemble of events}

If our estimates for the one-dimensional cumulative distributions $P(<x)$ are unbiased and if $x_*$ is a random variable
consistent with the prior, then $P(x_*)$ should be a uniformly-distributed random variable.   To test this hypothesis,
we use the one-dimensional posteriors provided by the MDC.
%% we perform repeated simulations, where each injected event was drawn from our prior:
%% \begin{itemize}
%% \item 2015 BNS MDC:   We selected \nEventsMDC{} events from the 2015 BNS MDC, identified by the \gstlal{} pipeline.  While all events
%%   have two-dimensional skymaps produced by \BS{}, the analysis presented below did \textbf{not} use two-dimensional skymaps.

%% While the NS-NS binaries in the 2015 MDC had generic spins, our parameter estimation model assumed zero spin.
%% \end{itemize}

% POINT: pp plots for an ensemble of events
For each parameter $x$, each colored curve in Figure  \ref{fig:pp:2015Ensemble} is  the fraction of events with
estimated cumulative probability $P(<x_*)$ at the injected parameter value $x_*$.  
Specifically, if $P(x_{*q})$ are the sorted cumulative probabilities for the $q=1\ldots n$ events with
$P(x_{*1})<P(x_{*2})$, then the points on the plot are $\{P(x_{*,q}),q/n\}$.  
%

\begin{figure}
%\includegraphics[width=\columnwidth]{../Figures/2015_BNS_MDC_pat_and_chris_pp_plot}   % Original runs
\includegraphics[width=\columnwidth]{../Figures/v1_2015_BNS_MDC_skysampling_pp_plot}  % Bayestar as prior and sampling prior
\caption{\label{fig:pp:2015Ensemble}\textbf{PP plot for ensemble of XXX NS-NS events}: \emph{Top panel}: For \textbf{X} randomly-selected NS-NS binaries, a plot of
  the cumulative distribution of $P_\theta(\theta_k)$ for each extrinsic variable $\theta=d,RA,DEC,\iota,\psi,\phi_{\rm
    orb}$.  \editremark{Beware: currently non-skymap code}
\emph{Bottom panel}: The sky area associated with higher-probability pixels than the true sky position of the source. \textbf{CREATE}
}
\end{figure}

{\color{blue} Integrate: The event selection process outlined \ref{sec:BNS_2015_MDC} introduces a small selection (Malmquist) bias, described in Figure \ref{fig:SearchSelection}, which slightly disfavors edge-on binaries relative to our prior.  Our parameter estimation strategy does not account for selection biases; for a sufficiently large ensemble of events, small deviations between the statistical properties of our posteriors and the ensemble are expected. }

\begin{figure}
\includegraphics[width=\columnwidth]{../Figures/fig-mma-manual-2015MDC-SelectedEvents-DistanceCumulative}
\includegraphics[width=\columnwidth]{../Figures/fig-mma-manual-2015MDC-SelectedEvents-CosIotaCumulative}
\caption{\label{fig:SearchSelection}\textbf{Selection biases}: For the 450 (\textbf{currently 120}) events used in our followup study, the cumulative distribution of $d^3$
  and $\cos \iota$.  In the absence of selection biases and in the limit of many samples, these two parameters should be
  uniformly distributed; small deviations away from uniformity reflect selection biases and sampling error.
}
\end{figure}


\subsection{Scaling}

For a quasicircular compact binary, it is well-known that the time to coalescence from a given GW frequency scales as
$t(f) \propto f^{-8/3}$. As the sensitivity of detectors improves at low frequencies, this requires the use of considerably
longer waveforms for detection and parameter estimation. For example, the initial LIGO detectors were sensitive down
to 40 Hz, while the advanced LIGO detectors could be sensitive down to 10 Hz. To cover extra low frequency
portion would require waveforms that are $\approx 40$ times longer.

Traditional Bayesian parameter estimation is computationally limited by waveform generation and
likelihood evaluations. Both of these are linearly proportional to waveform length. Note that the likelihood evaluations 
involve computing an inner product as in Eq.~\ref{eq:InnerProduct}, 
which is approximated as a finite sum. The number of points in the sum is determined by the length of the waveform 
and data being analyzed, which is why the cost of likelihood evaluations scales with waveform length.
Therefore, one would expect the cost of Bayesian parameter estimation using a seismic cutoff of $f_{\rm min} = 10$ Hz
to be roughly 40 times more expensive than the same analysis using $f_{\rm min} = 40$ Hz.

The method proposed here is not computationally limited by waveform generation. Recall that for each point
in the intrinsic parameter space we compute the waveform and the inner products between the various modes
and the data (the assorted $Q_{k,lm}$, $U_{k,lm,l'm'}$, $V_{k,lm,l'm'}$) only once. We then integrate over the 
extrinsic parameters, which involves evaluating $F_+ + i F_\times$ and the $Y^{(-2)}_{lm}$'s for 
different values of the extrinsic parameters.
While generating the waveform and computing inner products does scale with waveform duration, 
this cost is insignificant (even for $f_{\rm min}=10$ Hz) compared to the integration over extrinsic parameters,
which is wholly independent of waveform duration.
Therefore, the cost of our method increases only a little as $f_{\rm min}$ is decreased, in contrast to the sharp
increase that occurs for waveform-limited techniques such as traditional Bayesian parameter estimation.
See Fig.~\ref{fig:fmin_scaling}.

\ForInternalReference{
* scaling versus number of harmonics used.  (Aside on truncating harmonics with trivial content...not used presently)
}


\ForInternalReference{
\begin{table}
\begin{tabular}{lll}
$f_{\rm low}$ & $t_{\rm wave}$ & $T_{\rm wall}$ \\\hline
10 & & \\
25 & & \\
30 & & \\
\end{tabular}
\caption{\textbf{Runtime versus starting frequency}: Waveform duration $t_{\rm wave}$ and wallclock time $T_{\rm wall}$  needed to evaluate $L_{\rm red}=\int L p d\theta$
  for one set of intrinsic parameters $\lambda$ versus starting frequency $f_{|rm low}$, for a $m_1,m_2=\textbf{XXX}$
  nonspinning black hole binary.  Waveforms were generated using the  \texttt{TaylorT1} and \texttt{EOBNRv2} time-domain
  codes, respectively.  The
  computational cost does not depend significantly on waveform duration for starting frequencies of interest. 
}
\end{table}
}

\begin{figure}
\includegraphics[width=\columnwidth]{../Figures/fig-manual-RuntimeScalingVsFmin.png}
\caption{\label{fig:fmin_scaling}\textbf{Scaling versus frequency}: Points show the runtime of our parameter estimation strategy as a function
  of the minimum frequency $f_{\rm min}$.  For comparison, the solid curve shows the scaling $\propto f_{\rm
    min}^{-8/3}$ expected if our runtime was proportional to the waveform duration (e.g., runtime proportional to the
  number of time samples). 
% INTERNAL REF: coinc_id_17494 was used here.
 Waveforms were generated using the standard \texttt{TaylorT1} time-domain code, with $m_1=1.55 M_\odot$ and $m_2=1.23 M_\odot$. 
}
\end{figure}



\section{Significance and Comparison with related work }
\label{sec:Discussion}

% POINT: Low latency
We have demonstrated a viable strategy for low-latency parameter estimation for long-duration binary neutron star signals, using
a production environment and currently-available computing configuration.   
%
Like low-latency sky localization 
\cite{gwastro-skyloc-Sidery2013,LIGO-2013-WhitePaper-CoordinatedEMObserving}, already provided with reasonable accuracy
by approximate methods like \BS{}, low-latency parameter estimation will enable rapid electromagnetic followup and interpretation of
candidate gravitational wave events. 


% POINT: Fast, enables large-scale tests, to get background and to perform other tests
%
More broadly, by dramatically decreasing the turnaround time for each analysis and by scaling to harness all available
resources efficiently, our strategy may significantly increase size and scope of parameter estimation investigations.
%
Indeed, because our code converges \emph{even more quickly} in the absence of a signal, our approach could be applied to
systematically follow up timeslides, providing a systematic understanding of the additional detection confidence
parameter estimation could provide. 
%


% POINT: Technical reasons why our procedure will be useful for environments with limited computing, 
%   like Virgo's clusters, etc
Finally, because our implementation has bounded runtime -- one hour is the \emph{worst} case -- we know what  resources
will be needed to analyze a given NS-NS binary. Moreover, the parallel algorithm can exploit all available computing
resources, without need for communication or coordination between jobs, allowing it to operate in computing environments
with tightly constrained wallclock time. 


\subsection{Reduced-order quadrature}
Reduced-order quadrature methods provide an efficient representation of a waveform family and any inner products against
it.   Other authors have recently proposed prefiltering the data against the reduced-order basis
\cite{gw-astro-ReducedOrderQuadraturePE-TiglioEtAl2014}, achieving significant speedup.  For example, using TaylorF2
templates, \cite{gw-astro-ReducedOrderQuadraturePE-TiglioEtAl2014} claim runtimes of order 1 hour, comparable to our
end-to-end time in the high-precision configuration described above.

% POINT
Our strategy and reduced-order modeling achive a similar speedup for qualitatively similar reasons: both strategies
prefilter the data.   In our algorithm, at each mass point, the data is prefiltered  against a
set of $h_{lm}$, then efficiently reconstruct the likelihood for generic source orientations and distances.  
%
By integrating the likelihood at each mass point over all extrinsic parameters,  we are dominated by extrinsic-parameter
sampling and  hence not limited by waveform generation.

% POINT: 
In the short term, reduced-order methods require further tuning and development, to calibrate their interpolation in
targeted mass regions and with specific PSDs.
Moreover, as the starting frequency is reduced, reduced-order methods do require additional basis vectors, increasing
their operation count and computational cost as $f_{\rm low}$ is reduced.
%
  By contrast,
our algorithm can be immediately applied to any noise curve and existing time-domain model that provides $h_{lm}$,  at any mass,
including EOBNRv2HM \cite{gw-astro-EOBNR-Calibrated-2009} and SEOB \cite{gw-astro-EOBspin-Tarrachini2012}.  Minimal
updates are  needed in \texttt{lalsimulation} to provide $h_{lm}$ (e.g., $h_{22}$) for most
other existing time- and frequency-domain waveform approximants.  
%
%
Finally, by construction our dominant operation count cost is  independent of the waveform's length (or number of basis
vectors).  Hence, unlike reduced order methods, our code will run in nearly same amount of time now and with full
aLIGO-scale instruments with $f_{\rm low}\simeq 10\unit{Hz}$.  


\subsection{Alternative parallelization schemes}
% POINT:
Any strategy that can compute evidence reliably over a sub-volume or hypersurface in parameter space
can be efficiently parallelized.   In the strategy described here, we parallelized via accurate, independent extrinsic marginalization.  
%
Other strategies are \textbf{probably being developed; we should ask}.


\section{Conclusions}
\label{sec:Conclude}

%Conclusions go here.
%% MAJOR POINT: Summarize paper
% POINT: Rationale, again
In the era of multimessenger astronomy, rapid and robust inference about candidate compact binary gravitational wave
events will be a critical science product for LIGO, as  colleagues with other instruments  perform followup and
coincident observations \cite{LIGO-2013-WhitePaper-CoordinatedEMObserving}.  
% POINT: Summary: code (see abstract)
Motivated by the need for speed, we have introduced an alternative, highly-parallelizable architecture for compact
binary parameter estimation.   
%   
First, by using a mode decomposition  ($h_{lm}$) to represent each physically distinct source and by
prefiltering the data against those modes, we can efficiently evaluate the likelihood for generic source positions and
orientations, independent of waveform length or generation time.   
% 
Second, by integrating over all observer-dependent (extrinsic) parameters and by using a purely Monte Carlo
integration strategy, we can efficiently \emph{parallelize} our calculation over the intrinsic and extrinsic space.  
%
Third, to target specific intrinsic (and extrinsic) parameters for further investigation, we ingest information provided
by the searches and \BS{}: the trigger masses and estimated sky position.  
% POINT: Conclusion: Code is fast
Using standard time-domain waveforms in a production environment, we can already fully process one event in less than 1 hour, using roughly $1000$ cores in parallel,
producing posteriors and evidence with reproducibly small statistical errors (i.e., $\lesssim 1\%$ for both).  
%
As our code has bounded runtime, almost independent of the starting frequency for the signal, a nearly-unchanged strategy could
 estimate NS-NS parameters in the aLIGO era.  

As an additional advantage,  our approach  produces both a posterior distribution and  a very accurate likelihood versus extrinsic parameters.  Unlike MCMC and nested-sampling codes, we can trivially \emph{reweight} our results, allowing the user to
reprocess the posterior using any user-specified intrinsic-parameter prior.  
%
As a result, our approach \emph{also} trivially enables several other calculations of great practical interest:
reanalysis given alternative astrophysical priors;  simutaneous analysis of multiple events, adapting the ``prior''
intrinsic (mass, tide \cite{2014arXiv1402.5156W,2013PhRvD..88d4042R}, cosmological parameter \cite{2010ApJ...725..496N,2012PhRvD..85b3535T,2012PhRvL.108i1101M}) distribution to reproduce multiple observations; and simultaneous independent constraints from
multimessenger observations. 

%% MAJOR POINT: Why this is a big deal
% POINT: Implications: Fast performance on BNS, with minimal additional cost at low frequency
%       Identified in LIGO white paper/PE technical page as high-priority



%% MAJOR POINT: Explicit dimensionality as weakness. 
%   - demonstrate being aware of other projects in PE
%   - demonstrate team player : useful for targeted goals
%   - but also demonstrate we have a plan -- this is *not* a dead end, and we *can* work with reduced-order modeling
While the alternative architecture proposed here is efficient and highly parallelizable over extrinsic parameters,
all \emph{other}  parameters are (currently) suboptimally explored.  
For example, the concrete algorithm described and implemented here adopts a \emph{fixed, low-resolution} grid to sample
 two mass dimensions.  
%
While the method described here should generalize to a few additional dimensions, substantial computational resources or
additional architectural changes (and fast waveform generation) would be needed to apply a similar technique to many higher-dimensional problems being addressed with Markov
Chain or nested sampling codes, including testing GR; self-consistent data, noise,  and glitch models;  and self-consistent
electromagnetic and gravitational wave parameter estimation.   
%% MAJOR POINT: Connect to other active projects  -- we are not an island
% POINT: Big picture
That said,  several methods have been proposed for rapid waveform interpolation, including SVD and reduced-order
methods.  In the long run, we  anticipate being able to perform Monte Carlo integration over intrinsic dimensions as
well, without being forced  to adopt the relatively ad-hoc intrinsic/extrinsic split presented here.  
%By contrast, by accelerating waveform generation, other strategies like reduced-order-quadrature may eventually accelerate
%conventional strategies by a comparable factor, allowing rapid analysis of these and other problems.    
%

% POINT: Playing well with others
To provide a complete proof-of-principle illustration of our algorithm, we developed an independent production-ready
code.  That said,  the standard \textsc{lalinference} parameter estimation library in general and existing parameter
estimation codes (\textsc{lalinference\_mcmc} and \textsc{lalinference\_nest}) could  implement some or all of the
low-level and algorithmic changes we describe.  For example, MCMC codes could implement our $h_{lm}$-based likelihood,
then de-facto
 marginalize over all extrinsic parameters by strongly favoring jumps at fixed intrinsic parameters ($\lambda$).
Any  implementation which provides accurate marginalized probabilities (e.g., $L_{\rm red}$) can be parallelized across parameter
space.  
%
% POINT: Recognize
%
We hope  that by combining paradigms and consolidating code, future parameter estimation
strategies can reach extremely low latencies, ideally of order a few minutes, when advanced detectors reach design sensitivity. 


\appendix

\section{Notation, definitions, and equations}
\subsection{Definitions}
\begin{itemize}
\item $\lambda$ : intrinsic coordinate, including masses and spins.

\item $\theta$ : extrinsic coordinate, including $d,RA,DEC,\iota,\psi_L,t,\phi_{\rm orb}$

\item $p_s(\theta)$: (joint) sampling prior in extrinsic dimensions

\item $p(\theta)$ : prior on extrinsic parameters

\item $\Like(\lambda,\theta)$ : likelihood.  In terms of individual detector strains $H_k$ and power spectra, provided by
\begin{eqnarray}
\ln L &\equiv \sum_k \ln L_k  = \ln L_{\rm model} + \ln L_{\rm data} \\
\ln L_{\rm model} &\equiv -\frac{1}{2} \sum_k \qmstateproduct{H_k}{H_k}_k  \\
\ln L_{\rm data} &\equiv  \sum_k \text{Re} \qmstateproduct{H_k}{\hat{H}_k}_k 
\end{eqnarray}

\item $Z(\lambda) \equiv L_{\rm red}(\lambda,\theta)$ : reduced or integrated likelihood, derived from $L$ via
\begin{eqnarray}
Z(\lambda) = L_{\rm red}(\lambda) = \int d\theta \; p(\theta) L(\lambda,\theta)
\end{eqnarray}

\item 
$w=Lp/p_s$ : weight

\item 
$n_{\rm eff}$ : ``effective number of samples''
\begin{eqnarray}
n_{\rm eff} \equiv  \frac{\sum_k w_k}{\text{max}_k w_k}
\end{eqnarray}

\item 
$h(t|\lambda,x)=h_+-i h_\times$ : complex gravitational wave strain

\item 
$h_{lm}(t)$: coefficients of a spin-weighted spherical harmonic decomposition
\begin{eqnarray}
\label{eq:def:hSpinWeightEmissionDirection}
h(t|\lambda,\theta) = \sum_{lm} h_{lm}(t|\lambda) e^{-2i\psi}\Y{-2}_{lm}(\theta_{JN}\phi_{JN})
\end{eqnarray}

\item 
$\tilde{h}(f)$ : two-sided Fourier transform of the complex function $h(t)$
\begin{eqnarray}
h(t) = \int_{-\infty}^{\infty} \frac{d \omega}{2\pi} \; e^{-i\omega t} \tilde{h}(\omega) 
\end{eqnarray}

%% \item
%% ${\cal I}$ : complex conjugation in time.  Provided to avoid confusion with $\tilde{h}^*$.  \textbf{Hopefully we won't
%%   need it.}


\item 
$\vec{x}_k$ : Position of the $k$th detector

\item 
$F_{+}$, $F_{\times},F$ : detector response function  to the $+,\times$ polarizations for sources visible in the
  $\hat{n}$ direction relative to detector
\begin{eqnarray}
F(\hat{n}) = F_+(\hat{n}) +i F_\times(\hat{n})
\end{eqnarray}

\item 
$\hat{H}_k$ : measured strain in  the $k$th detector

\item 
$H_k$ : strain response of the $k$th detector to an incident strain $h$
\begin{align}
H_k(t) &=F_{+,k}(t) h_+(t-\vec{x}_k(t)\cdot \hat{k}) + F_\times(t) h_\times(t-\vec{x}_k(t)\cdot \hat{k}) \\
 &=  \frac{F h(t-\vec{x}_k\cdot \hat{k}) }{2} + \frac{F^*h^*(t-\vec{x}_k\cdot \hat{k})}{2}
\end{align}

\item 
$S_k$ : noise power spectrum for the $k$th detector

\item 
$\qmstateproduct{a}{b}_k$ : complex-valued inner product defined by the $k$th detector's noise power spectrum:
\begin{eqnarray}
\qmstateproduct{a}{b}_k \equiv 2 \int_{-\infty}^{\infty} df \frac{[\tilde{a}(f)]^*\tilde{b}(f)}{S_h(|f|) }
\end{eqnarray}

\end{itemize}


\section{Discrete data handling}
\label{ap:DiscreteData}
% COMPARISON: Findchirp, http://arxiv.org/pdf/gr-qc/0509116v2.pdf

In the text we describe the core algorithm using continuous time- and frequency notation, omitting most practical
details involved in operating on discrete timeseries.  In this appendix, we describe in detail the operations we perform
on discretely-sampled detector timeseries $\hat{H}_k$  and  waveform modes $h_{lm}(t)$ to calculate the likelihood
provided in the text.

\subsection{Discrete filtering}
% REQUIRED
%    - h: How we got it, and the fourier transform.
%    - frequency limits
%    - S_h and inverse spectrum truncation

% POINT: Data selection and time windows
Based on a proposed GPS geocenter event time $t_*$ and intrinsic parameters $\lambda$, we operate on all discretely-sampled detector
data $\hat{H}_k(t)$  in each instrument $k$ for the following GPS time interval:
\begin{align}
t &\in [-T_{\rm seg},T_{\rm seg}]+t_{*} \\
T_{\rm sec} &= T_h+T_{\rm spec} +T_{\rm safety} \\
T_{\rm safety} &= 2\unit{s} \\
 T_{trunc} &= 8 \unit{s} \\
T_h &= \text{waveform duration at }\lambda
\end{align}
In this expression, we compute the waveform duration $T_h$ by evaluating the simulated waveform from a starting
frequency $f_{\rm low}$ until merger. 

% POINT:
All waveform modes $h_{lm}$ are provided on a discretely-sampled time grid by \texttt{lalsimulation}.  
%
All waveform and detector-data ($\hat{H}$) timeseries are zero-padded to the next power of 2 before further operations are
performed.  
%
No windowing is performed on either the data or input waveform, either to smooth unphysical startup and termination
discontinuities in $h_{lm}(t)$ or to eliminate discontinuities between the starting and ending timesample in $\hat{H}(t)$
%
All detector and waveform data are sampled at 16384 Hz.  
%


% POINT: Two-sided fourier transform: hlmoff
The discretized fourier-transform of a complex-valued timeseries $h(t)$ on $N$  values $t_j=j\Delta t+t_0$ is
implemented as usual by the forward complex FFT:
\begin{eqnarray}
\tilde{h}(k) = \Delta t  \sum_k e^{2\pi i j k} e^{2\pi i \Delta f k t_o} h(t_j)
\end{eqnarray}
where $k=0\ldots N-1$.  Bins $k\le N/2$ correspond to positive frequencies and $k>N/2$ correspond to  negative
frequencies.    As all data segments  have the same sampling rate and cover the same time interval, we henceforth set $t_0=0$.  
% 

% POINT: PSD estimate, resampling, and inverse spectrumtruncation
We rely on other codes to assess and report on detector noise near the event.   From the recorded discrete noise power
spectrum $S_0(f)$, we construct a discrete Fourier-domain inverse spectrum filter $K(f)$ on a targeted frequency interval $|f|
\in [f_{\rm min}, f_{\rm max}]$ with frequency spacing $\Delta f$ by 
% lalsimutils.get_psd_series_from_xmldoc
% lalsimutils.resample_psd_series
(a) interpolating $1/S_0(f)$ onto a discrete frequency grid, as $K_0(f)$; 
% lalsimutils.ComplexIP : inv_spec_trunc_Q=True
(b) coarse-graining the power spectrum by performing \emph{inverse spectrum truncation} \cite{2012PhRvD..85l2006A} to construct a filter with bounded duration $T$ from $K_0$:
\begin{eqnarray}
 K =  |{\cal F} [\Theta_{T_{\rm spec}} \times {\cal F}^{-1}[ \sqrt{K_0} ]]|^2
\end{eqnarray}
where ${\cal F}$ is the fourier transform; $\Theta_T(t)$ is a step function in time which is nonzero only for $t\in[-T/2,T/2]$; and $T_{\rm spec}=8\unit{s}$ is a
constant time interval. 
%
In the frequency interval $|f|\in[f_{\rm min},f_{\rm max}]$, the discrete inverse filter $K(f)$ nearly agrees with the
input inverse spectrum $1/S_0(f)$.  Unlike the original filter, however, the inverse-spectrum-truncated filter has
support for all frequencies $f\ne 0,f_{\rm Nyq}$.     
%
To be concrete, inverse spectrum truncation is implemented precisely as previously  \cite{2012PhRvD..85l2006A}, using
real (one-sided) discrete fourier transforms: 
%
% lalsimutils.ComplexIP : inv_spec_trunc_Q=True
we populate an array of length $1+f_{\rm Nyq}/\Delta F$ with values $K_{0,q} =1/S_o(q \Delta f)$ from $q\Delta f$ between $f_{\rm min}$ and $f_{\rm max}$; 
we perform one-sided fourier transform of $\sqrt{K_{0}}$, populating a complex array of length $N=2 f_{\rm Nyq}/\Delta f$; 
we set all bins with $q\in [N_{\rm spec}, N-N_{\rm spec}]$ to zero, where $N_{\rm spec} = \floor*{T_{\rm spec}/\Delta
  T}$; 
and  finally we inverse-fourier transform and square to construct a one-sided real-valued array $K(q\Delta f)$.  
%
The two-sided filter is constructed by mirroring the original one-sided array $K$ about $f=0$, using the same packing
scheme we adopted for two-sided fourier transforms.  
%  - I don't want to discuss how lalsuite packs its FFT arrays

% POINT: Discrete filter for inner products
The inner product arrays $U_{k,lm,l'm'}$ and $V_{k,lm,l'm'}$ are constructed using this discrete filter by a discrete
approximation to the integral.  For example, the first array is evaluated as
\begin{eqnarray}
U_{k,lm,l',m'}(\tau_q) = 2 \sum_s \Delta f  K_k(f_s) \tilde{h}_{lm}^*(f_s) \tilde{h}_{l'm'}(f_s)
\end{eqnarray}
The sum includes all frequency bins, corresponding to both positive and negative frequency.  % Specifically, we do not re-impose any constraint on $f$.


% POINT: Discrete filter output
The $Q_{k,lm}(t)$ are the output of continuously filtering each detector's data against the $h_{lm}$ mode.  Using the
discrete arrays representing the fourier-domain modes $\tilde{h}_{lm}(f_q)$, data $\tilde{\hat{H}}_k(f_q)$, and inverse-noise-spectrum  $K_k(f_q)$, we estimate the
filter output via an inverse fourier transform:
\begin{eqnarray}
Q_{k,lm}(\tau_q) = 2 \sum_s \Delta f e^{+2\pi i \tau_q f_s} K_k(f_s) \tilde{h}^*_{lm}(f_s) \tilde{\hat{H}}_k(f_q)
\end{eqnarray}
%
As needed, we evaluate $Q_{lm,k}(\tau)$ for arbitrary $\tau$ by  cubic spline interpolation.  


% POINT: Finite time window?  Provided in text


\subsection{Discrete time marginalization with and without interpolation}

% POINT: What time marginalization is
The likelihood can be efficiently marginalized in time by introducing a discrete grid:
\begin{eqnarray}
\int_{t_*-T_{\rm window}/2 }^{t_*+T_{\rm window}/2}L(t) \frac{dt}{T_{\rm window}} \simeq \Delta t \sum_q e^{\ln L(t_q)}
\end{eqnarray}
where $L(t) $ is the likelihood versus time, all other parameters fixed.   
%

% POINT
We have implemented time marginalization using two approximations for the likelihood $L(t)$ versus time.  In one,  the  functions
$Q_{k,lm}(\tau)$ are evaluated using cubic spline interpolation; in the other, a method equivalent to nearest-neighbor interpolation.  For
the 16kHz data rate and signal amplitudes used, the two methods agree.   Because nearest-neighbor interpolation
corresponds to using a shifted version of the  discrete output $Q_{k,lm}(\tau_s)$ of the discrete filter, the latter
implementation is slightly faster.   
%
Unless explicitly stated otherwise, we adopt the latter method to produce all results.  


\ForInternalReference{
\section{[Internal use] : Weaknesses people may ask about at LVC}

* Prior on sky is skymap

* Discrete sky grid: biases? What about high SNR? (A: just use adapted resolution; not ready yet)

* Uniform weighting

* Not discarding values after burnin

* Adaptation algorithm: why parameters chosen?

* Computational cost: is this really favorable in CPU-hours?  Are we wasting CPU hours to achieve low-latency that's not needed?

** A: Not clear our overall cost is higher...we may be more accurate than 1000 samples, the usual metric for MCMC...we need a clear comparison

* Rotation of the earth

\subsection{Runtime}

* Might be ram-limited in python below 10-ish Hz (multiple harmonics and IFOs, sampled at 16kHz for 30 min)
%	1byte/sample * 16384*30min *60s/min ~ 28 Mb 

\subsection{Applications}

\textbf{These need to be mentioned on the CBC project page}

* Einstein at home scaling?

* GRB project with Alex (already on table)

* [partial] Port likelihood into lalinference, see if they can get any improvement with it, mainly for precessing systems.

\section{[Internal use] Postprocessing details}
Notes, possibly for integration with main text
\begin{widetext}

\noindent \textbf{Doing the integral: Polar coordinates for linear spoked integration}: Our spoked mass grid is
naturally interpolated in polar coordinates $r,\theta$.  Keeping in mind the change-of-coordinates to align the
principal axes of our error ellipsoid with the data, we can evaluate the evidence integral as 
\begin{align}
(\mc,\eta)_a &= [\sqrt{\Gamma}]_{ab}  (r\cos \theta,r\sin \theta)_b \\
\int d\mc \eta p(\mc,\eta) \LikeRed(\mc,\eta) &= \int |\Gamma| r d\theta \; p(\mc(r,\theta),\eta(r, \theta)) \LikeRed(\ldots)
\end{align}
where the matrix square root of the Fisher matrix is used to change coordinates to (scaled) principal axes of the ellipse:
\[
\sqrt{\Gamma}_{ab} = \sqrt{\gamma}_1 \hat{v}_{1,a}\otimes v_{1,a} + \sqrt{\gamma_2} v_{2b}\otimes v_{2b}
\]
where $\{\gamma_k,v_k\}$ is the eigensystem of $\Gamma$.

\end{widetext}


\section{[Internal use] Monte Carlo integration}

\subsection{Skymaps}
\noindent \textbf{[Skymaps]  Discretizing a volume} We integrate over the sky by translating a continuous distribution into
discrete, equal-area pixels.  Specifically, we start with the abstract process of breaking an integral over a volume
$\Omega$ in the sky coordinates ($\gamma$)  up into nonoveralapping
volumes $A_n$ with $\cup A_n = \Omega$  , where the extrinsic volumes are small enough that we can treat any expected
function (e.g., $L,p$) can be approximated as nearly independent of $\gamma$ in $A_n$:
\begin{widetext}
\begin{align}
\LikeRed(\lambda)
& = \int_{V\times\Omega}\Like(\lambda,\theta,\gamma)p(\theta)p(\gamma) d \theta d\gamma 
%\\& 
=  \sum_{n=1}^N \int_{A_n} [\int_V  L(\gamma,\theta,\gamma) p(\theta) ] p(\gamma)d\gamma 
%\\ & 
 \simeq  \sum_{n=1}^N  \left [\int_V  L(\gamma,\theta,\gamma_n) p(\theta) \right] P_n 
\end{align}\end{widetext}
where $P_n \equiv \int_{A_n} d\gamma p(\gamma)$ is the prior probability for the volume $A_n$, and $\gamma_n$ is a
representative point in $A_n$.   Note
\begin{eqnarray}
\sum_n P_n =1
\end{eqnarray}
We are \emph{particularlly} interested in uniform distributions (equal-area tilings of the sky), so 
\[
P_n = \frac{1}{N} \qquad   \text{Area}(A_n) = \frac{4\pi}{N}
\]

\noindent \emph{Monte Carlo integration with discrete volumes}: As above, but apply to Monte Carlo integration, with a
sampling prior $p_s(\theta)p_s(\gamma)$ in the sky $\gamma$ and all other extrinsic variables $\theta$:
\begin{align}
\LikeRed(\lambda)
& \simeq  \sum_{n=1}^N \int_{A_n} \left [\int_V  L(\gamma,\theta,\gamma_n) p(\theta)p(\gamma) \right]
\end{align}
Proceed as above, but approximate $p(\gamma_n)/p_s(\gamma_n) \simeq P_n/P_{s,n}$, to find
\begin{align}
\LikeRed(\lambda)
& =\sum_n \left[\int_{V}\Like(\lambda,\theta,\gamma_n)\frac{p(\theta)}{p_s(\theta)}p_s(\theta)d\theta \right] \frac{ P_n }{P_{s,n}} P_{s,n}
\end{align}
Now imagine performing a Monte Carlo integral in this form.   Monte Carlo integration corresponds to selecting a random
integer $n$ and a random sample $\theta$ drawn from $p_s(\theta)$.  Indexing the $N$ choices as $n_{\alpha}$ and
$\theta_\alpha$, we find
\begin{align}
\LikeRed(\lambda)
& = \sum_{\alpha}\frac{1}{N}
\Like(\lambda,\theta_{\alpha},\gamma_{n,\alpha})\frac{p(\theta_\alpha)}{p_s(\theta_\alpha)} \frac{ P_{n_\alpha}
}{P_{s,n}} 
\end{align}
In other words, when performing a Monte Carlo integral over a discrete subset (e.g., skymap), we use the
\emph{integrated} probabilities $P_n$ as our ``PDF''.

\emph{Concrete discrete random sampling}: For a discrete uniform equal-area tiling, we will use a
\textbf{discrete} probability distribution, not a continuous one.   Specifically, our code 
\begin{itemize}
\item \emph{Prior PDF}: Since we always assume uniform on the sky,  provide $P_n=1/N$ at the $n$th
point (i.e., given the sky position of the $n$th tile), \emph{not} $p(\gamma_n)$.

\item \emph{Skymap sampling PDF}: Bayestar provides normalized weighted $s_{s,n}$ already.  
%% After fixing a
%%   \textbf{normalization issue} (below), we use
%% \[
%% P_{s,n} = s_n \frac{N'}{N}
%% \]
%% where $N,N'$ are defined below.

\item \emph{Skymap random samples via CDF inverse}: We generate random integers $n$ using random numbers on $[0,1]$, finding the
  closest integer $n$ to the  cumulative of $P_n$; see the text.
\end{itemize}




\noindent \textbf{[Skymaps] Finite-area skymaps and truncation}: We will use skymaps with
$p_s(\gamma)=0$  in a volume $U$ where $p(\gamma)\ne 0$.  With restricted support, we can't integrate
integrate generic functions $L$ with $L>0$ with $p_s=0$.  To be concrete, if we split up the integral over the sky ($\Omega$) into $U$ and
$\Omega-U$, we are ignoring the contribution from $\Omega-U$:
\begin{align}
\LikeRed(\lambda) &= \int_U d\gamma p_s(\gamma) 
 [\int d\theta p_s(\theta)
  \frac{p(\theta)p(\gamma)}{p_s(\theta)p_s(\gamma)}\Like] 
  \nonumber \\ &
 + \int_{\Omega-U}d\gamma p(\gamma) p(\theta)\Like 
\nonumber \\ &
\simeq \int_U d\gamma p_s(\gamma) [\int d\theta p_s(\theta)
  \frac{p(\theta)p(\gamma)}{p_s(\theta)p_s(\gamma)}\Like] 
\end{align}
The second step introduces systematic error, assumed small.

The second integral is a conventional Monte Carlo integral over a region $U$ using normalized sampling distributions $p_s$ and
can be approximated in the usual way.

Note the ratio of $p/p_s$ \emph{automatically} includes any ``normalization factor'' associated with only drawing
samples from $U$ rather than from the whole volume $V$

%% , rescaling the total number of samples $N_s$ drawn from $p_s$ by the
%% ratio of volumes of the prior and of $p_s$ over that region $U$
%% \begin{eqnarray}
%% N_{\rm corr} = N / \int_U p_s(\gamma) d\gamma
%% \end{eqnarray}
%% We will apply this correction to the returned values of $p_s$, returning ``rescaled'' probabilities 
%% % mcsampler.py  : self._renorm  in pseudo_pdf  and self._expand_valid_points.
%% \begin{eqnarray}
%% p_s \rightarrow \frac{p_s}{\int_U p(\gamma)  d\gamma}
%% \end{eqnarray}
%% Specifically,  if $s_n$ are the  skymap weights reported by \BS, normalized so $\sum_n s_n=1$, we use an ``effective''
%% sampling ``probability'' $P_{s,n}$ of
%% \begin{eqnarray}
%% P_{s,n} = s_n \frac{N'}{N}
%% \end{eqnarray}
%% where $N'$ is the number of nonzero samples in $s$ and $N$ is the number of pixels in $s$.

\noindent \textbf{[Skymaps] Subtle implementation issues: cutoffs and lookup tables}: As described in the text, we use a
discretized lookup table.  As not described in the text, this table de facto \emph{discards} points with weights the
quantized probability.   For real skymaps, the choice of probability quantum $P_{\rm min}$ can be what determines
precisely what points are sampled and hence what part of $\int L p d\theta d\gamma$ are are not sampling \emph{at all}.

As a concrete example, with $n=12*64^2$ sky points, a uniform weighting gives each pixel $P_{n}
=1/n \simeq 2\times 10^{-5}$.  A qunatum of probability $P_{\rm min} $ should \emph{definitely} be smaller than this, to
allow the option of recovering a uniform skymap.
%
More broadly, a skymap with a peak and a wide tail can have that tail truncated if $P_{\rm min}$ is too small.  For
example, if $5\%$ of the skymap probability is distributed across a wide area $\Omega$ of the skymap, we need $P_{\rm
  min} < 0.05 (4\pi/\Omega)/n$, to be sure that area is sampled.  
%
For safety, ROS recommends adopting a small probability quantum $\frac{10}{N}<P_{\rm min} < 0.01/n$, where $n$ is the
number of sky pixels and $N$ is the number of iterations in ILE.\footnote{This choice ensures that when ILE is run, at
  least a few draws occur even from the lowest probability pixels.  Does this matter?  Principally for tests, or if
we expect the \BS{} skymap might be wrong.}  \textbf{This choice requires self-consistent use of
  multiple code parameters, some currently hardcoded. }

Once chosen, the quantum of probability defines what pixes de facto have zero probability and are never sampled. 
%% \noindent \emph{Convergence issues and sampling bounds on  small probability quanta}: Small probability quanta introduce their own
%% potential problems.  The smaller the probability quantum, the more likely you are to have a pixel with $P_{s,n} \simeq
%% P_{\rm min}$.  On average, that pixel will be sampled every  $1/P_{\rm min}$ sky iterations -- a number that might be in
%% excess of all samples drawn in one ILE run!
Ideally, the likelihood will be zero in  low-probability pixels, no matter what $\lambda,\theta$ are, circumventing
any issues with undersampling and convergence rate.


\subsection{Monte Carlo integration}
\textbf{Non-uniform, variance-weighted weighting}: By default we combine results with different sampling priors treating
them equally (e.g., as if they have converged to the same sampling prior) when evaluating the integral at each mass
point.  

That doesn't minimize variance and can be particularly
ineffective when one ILE run has a problem.  Really, we should use variance-reducing weighting: if $x=ax_1 +b x_2$ with $a+b=1$ and
$\left<x_1\right> =\left<x_2\right> =\left<x\right>$, we can minimize the variance of $x$ by choosing weight $a$ to
minimize $a^2 \sigma_1^2 + (1-a)^2 \sigma_2^2$, so 
\begin{eqnarray}
x = \frac{\sigma_2^2}{\sigma_1^2+\sigma_2^2} x_1 +  \frac{\sigma_1^2}{\sigma_1^2+\sigma_2^2} x_2
\end{eqnarray}
A straightforward generalization exists with arbitrary numbers of weights:
\[
x = \frac{x_k/\sigma_k^2}{\sum_q 1/\sigma_q^2}
\]
We could and should implement this scheme when recombining results at each mass point. 
%% a^2 s1^2 + b^2 s2^2 + (1 - a - b)^2 s3^3
%% Solve[{D[%, a] == 0, D[% == 0, b]}, {a, b}]

\begin{widetext}
\noindent \textbf{Error estimates for $P(<x)$: Single-job}:  ILE reports on 1d cumulative distributions $P(<x)$ from
each job.  Both the numerator and denominator are Monte Carlo integrals using $N$ samples, allowing us to generate an
error estimate.  Keeping in mind the samples are independent and identically distributed, we can evaluate the standard
deviation of the numerator and denominator in the large-$N$ limit
\begin{align}
\hat{P} &= \frac{\sum_k w_k \theta(x-x_k)}{\sum_k w_k} \equiv \frac{ I(<x)}{I} \\
\left< \hat{P} \right> &\simeq \frac{1}{I} \sum_k \left<w_k \theta(x-x_k) \right>  \\
\left<I(<x)^2\right>& \equiv \left<[\sum_k w_k \theta(x-x_k)]^2\right> 
  = \sum_k[ \left<w_k^2 \theta(x-x_k)\right>  -\left<w_k\theta(x-x_k)\right>^2
  + \left<\sum_{k} w_k \theta(x-x_k)\right>^2]  \\
\text{estimate: }\sigma_{I(x)}^2 &= \sum_k w_k^2 \theta(x-x_k)  - I^2 P(x)^2 \\
\text{estimate: } \sigma_{I} &= \sum_k \left<w_k^2\right> - I^2 \\
\sigma_P^2 & \simeq \frac{\sum_k w_k^2 \theta(x-x_k)}{I^2} - P(x)^2 + (\text{value at } x_{\rm max})
\end{align}
\end{widetext}
where the last expression adds errors in quadrature from the numerator and denominator, assuming \emph{uncorrelated}
errors.    (In fact, the two are correlated \editremark{fixme: important for scaling the error as $P(1-P)$ })



\subsection{Marginalizing over parameter}

\noindent \textbf{Skymap marginalization: A proposal}: A discrete skymap usually has relatively concentrated support -- roughly
$10^3$ pixels or less, say.  Rather than perform Monte Carlo integration to approximate $\int P_k
\Like(\lambda,\theta_k)$ over the sky, we can \emph{directly} integrate, via a discrete operation:


\emph{Is this a good idea?}: This procedure requires evaluating the likelihood on  \emph{every pixel in the skymap}, so
is of considerable advantage only for very
narrowly targeted (3-IFO, high-SNR) sky localizations.  In such a narrowly-focused case, an adaptive integrator will almost certainly
discover the best-fitting single sky position.  That said, if we're worried about \emph{strongly-separated peaks}, this
procedure will provide an (expensive) backup.

\section{[Internal use] Results: Targeted studies}

\begin{itemize}
\item Single event, 100 noise realizations, plus comparison with Fisher matrix

\item Zero-spin 2015 MDC: [\textbf{Not started}]  For completeness, we propose to reproduce the 2015 MDC,  using
  injections with exactly zero spin.

\end{itemize}

\subsection{Detailed investigation of one event}
Sample run results: Single event, with full DAG, done multiple times (for complete consistency/reproducibility)

Sample run results: extrinsic parameters (for one and several noise realizations)

Sample run results: $L_{\rm red}(\mc,eta)$ (for one and several noise realizations). Expected accuracy at each mass
point.. Translating into  $p(m_1,m_2)$ using a uniform mass prior.

Comparison with Fisher and MCMC


\begin{figure}
\caption{\label{fig:TargetedEvent:LikelihoodVersusMchirpEta}\textbf{Targeted study: Posterior distribution
    versus component masses, for several noise realizations}: : The 90\% confidence interval derived from $L_{\rm red}$.  For
comparison, the prediction from a Fisher matrix is shown as a solid black curve.
 \textbf{PLACEHOLDER/INTENT}
}
\end{figure}


\section{[Internal use] Results: Additional production-environment investigations}

\begin{itemize}
\item 2016 BNS MDC: [\textbf{Not started}]

\end{itemize}

\section{[Internal use] Incomplete or not-yet-implemented investigations}

\subsection{Measures of convergence}

Measures to assess whether we're done
\begin{itemize}
\item * Integral: Classic MC error estimate (central limit); reproducibility across runs and via subsamples (e.g., chisquared)

\item * 1d posterior:  Monte carlo error (see elsewhere); $n_{\rm eff}$; $L^2$ or KL divergence between subsamples or
  across runs, compared to statistics.

\item * sampling distribution: similarly
\end{itemize}


\subsection{Semianalytic marginalization over distance}


\section{[Internal use] Future directions}

Not for distribution!

\subsection{Short GRBs}

Alex

\subsection{Adding dimensions}

Tides [Les]

Aligned spin

\subsection{Even more speed}

Because the precompute phase is \emph{much} faster than marginalizing over extrinsic parameters, we could accelerate
code performance in a number of ways, potentially enabling us to tackle higher dimensions

\noindent \textbf{Einstein at home}: Transmit the precomputed data (10ms of timeseries and a few scalars) to the
Einstein at home grid, giving each distant host (say) 20 mass points to handle.  With $10^5$ cores, that means roughly
$20\times10^5/10$ intrinsic points per day...enough to handle precessing spin faster than any other extant method.

\noindent \textbf{Interpolate $Q,U$}: Perform the precompute step several times, building up a grid for $Q$ and $U,V$,

\subsection{Tigher search integration}


\noindent \textbf{Search filters as input}
The searches provide filter outputs.  We ought to be able to use those as inputs to construct $Q_{k,lm}$ for any
$\lambda,l,m$, via a suitable lookup table.   This could completely eliminate the startup/precompute cost.



\noindent \textbf{As detection statistic?}: To what extent is this approach useful compared to coherent PTF?  We can
also do timeslides at a fixed sky location.


\section{[Internal use] Results: Targeted studies}

\begin{itemize}
\item Single event, 100 noise realizations, plus comparison with Fisher matrix

\item Zero-spin 2015 MDC: [\textbf{Not started}]  For completeness, we propose to reproduce the 2015 MDC,  using
  injections with exactly zero spin.

\end{itemize}

\section{[Internal use] Results: Targeted studies}

\begin{itemize}
\item Single event, 100 noise realizations, plus comparison with Fisher matrix

\item Zero-spin 2015 MDC: [\textbf{Not started}]  For completeness, we propose to reproduce the 2015 MDC,  using
  injections with exactly zero spin.

\end{itemize}

\subsection{Detailed investigation of one event}
Sample run results: Single event, with full DAG, done multiple times (for complete consistency/reproducibility)

Sample run results: extrinsic parameters (for one and several noise realizations)

Sample run results: $L_{\rm red}(\mc,eta)$ (for one and several noise realizations). Expected accuracy at each mass
point.. Translating into  $p(m_1,m_2)$ using a uniform mass prior.

Comparison with Fisher and MCMC



\section{[Internal use] Results: Additional production-environment investigations}

\begin{itemize}
\item 2016 BNS MDC: [\textbf{Not started}]

\end{itemize}

\section{[Internal use] Incomplete or not-yet-implemented investigations}

\subsection{Measures of convergence}

Measures to assess whether we're done
\begin{itemize}
\item * Integral: Classic MC error estimate (central limit); reproducibility across runs and via subsamples (e.g., chisquared)

\item * 1d posterior:  $n_{\rm eff}$; $L^2$ or KL divergence between subsamples or across runs

\item * sampling distribution: similarly
\end{itemize}


\subsection{Semianalytic marginalization over distance}


\section{[Internal use] Future directions}

Not for distribution!

\subsection{Short GRBs}


\section{[Internal use] Incomplete or not-yet-implemented investigations}

\subsection{Measures of convergence}

Measures to assess whether we're done
\begin{itemize}
\item * Integral: Classic MC error estimate (central limit); reproducibility across runs and via subsamples (e.g., chisquared)
\end{itemize}

\section{[Internal use] Future directions}

Not for distribution!

\subsection{Short GRBs}

Alex

\subsection{Adding dimensions}

Tides [Les]

Aligned spin

\subsection{Even more speed}

Because the precompute phase is \emph{much} faster than marginalizing over extrinsic parameters, we could accelerate

\section{[Internal use] Patches pending or desired}

Persons/proposing supporting  a change listed as [name].

\noindent \textbf{Next push}

* Bugfix on ILE MC variance

* Uniform on sky prior with bayestar skymaps

\noindent \textbf{Other important updates}

* [ROS] Add monte carlo error to $P(<x)$ estimate plots (optional) and description in text

*  [ROS]  $b_k \ne 0$ for any sky position (else complications arise re normalization).

* [ROS] Interpolation of $L$ vs masses using spokes (higher order) -- current plots are mathematica

\noindent \textbf{Minor code tweaks}

* [ROS] Larger nchunk

* [ROS] Change tempering exponent definition from $(Lp/p_s)^\beta$ to $L^\beta p/p_s$, to facilitate analysis

* [ROS] Variance weighting when recombining samples from different runs at the same mass point


\noindent \textbf{Infrastructure}

* [ROS] Change DAG (number of jobs, max iterations): current solution with skymap is overkill


* [ROS] For automated end-to-end tests, add \gstlal{} coinc generation and \BS{} skymaps to \texttt{stage\_injections}

** Approximate solution: use bayestar tool to generate ``fake'' coincs and skymaps from sims:
\href{https://www.lsc-group.phys.uwm.edu/ligovirgo/cbcnote/ParameterEstimationModelSelection/BAYESTARHowTo}{Bayestar docs}

* [ROS] Add posterior in mchirp, eta accounting for prior.

* [Patrick] Logging

* [Richard/Chris] Report hashtag in log and process\_params

* [ROS] Warnings about implicit cutoffs re skymaps (\texttt{min\_p} plus the \texttt{cdf} target, both of which define
regions to be ignored).



* [ROS] Modularized version that lets us change mass quickly (e.g., to loop over mass points; to test mass points
against known-good extrinsic parameters; etc)
%load all wights stored 
%loop and eval likelihood with stored samples only
%useful for exploring mass space using fixed extrinsic grid set by previous 
%only need a few samples to get an idea


\noindent \textbf{Exploration}

* [ROS] Region library (i.e. ellipsoids): test if in; perform integral (mcsampler); return points (grid/unstructured)

** mcsampler: return dump of params, so we can construct marginalized posteriors/PDFs using usual infrastructure

* [ROS] Refinement stage: a single job after a mass point, using the best-suited

\section{Methods  paper 2: Spin}

Adding the spin dimensions is important and worth a detailed study.


Very early; keep thoughts organized in outline, below


\begin{itemize}

  \item Introduction
  \item Revised methods
    \begin{itemize}
        \item  $h_{lm}$ in $J$ frame
       \item  Single spin: approximate harmonic method (treat each sideband $e^{im\alpha+im\gamma}h_{22}^{ROT}$ as a
         seperate filter, and $\beta$ as an extrinsic parameter.  De facto means adding only one additional dimension
         ($\kappa$) over aligned spin;
         all else extrinsic)
        \item Spin dynamics and dimensional reduction/reorganization.  ``PN resonance coordinates.''
        \item Role of prior for spin (6d vs 3d vs 2d prior for 2spin, 1spin, aligned spin) has effect
        \item New approximately extrinsic parameters (e.g., $\alpha$ as approximately extrinsic)
    \end{itemize}

 \item Discussion and sample results: Aligned spin
     
   * single event in detail: 3d posterior; evidence for spin vs zero spin

 \item Discussion and results: Single precessing spin

   * ``fixed $\beta$

   * single event in detail

 \item Discussion and results: Double spin

  * challenge of higher dimensions
\end{itemize}

}

\bibliography{overviewexport}
\end{document}
