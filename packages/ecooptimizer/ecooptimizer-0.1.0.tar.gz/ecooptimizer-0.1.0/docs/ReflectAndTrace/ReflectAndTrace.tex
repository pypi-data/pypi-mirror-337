\documentclass{article}

\usepackage{tabularx}
\usepackage{booktabs}

\title{Reflection and Traceability Report on \progname}

\author{\authname}

\date{}

%% Comments

\usepackage{color}

% \newif\ifcomments\commentstrue %displays comments
\newif\ifcomments\commentsfalse %so that comments do not display

\ifcomments
\newcommand{\authornote}[3]{\textcolor{#1}{[#3 ---#2]}}
\newcommand{\todo}[1]{\textcolor{red}{[TODO: #1]}}
\else
\newcommand{\authornote}[3]{}
\newcommand{\todo}[1]{}
\fi

\newcommand{\wss}[1]{\authornote{blue}{SS}{#1}} 
\newcommand{\plt}[1]{\authornote{magenta}{TPLT}{#1}} %For explanation of the template
\newcommand{\an}[1]{\authornote{cyan}{Author}{#1}}

/* Copyright (c) 2021 by InterSystems Corporation.
   Cambridge, Massachusetts, U.S.A.  All rights reserved.
   Confidential property of InterSystems Corporation. */

Include Ensemble

Class IOP.Common [ Abstract, ClassType = "", ProcedureBlock, System = 4 ]
{

/// One or more Classpaths (separated by '|' character) needed in addition to the ones configured in the Java Gateway Service
Property %classpaths As %String(MAXLEN = "");

Property %classname As %String(MAXLEN = "");

Property %module As %String(MAXLEN = "");

Property %settings As %String(MAXLEN = "");

/// Instance of class
Property %class As %SYS.Python;

/// Get Class
Method GetClass() As %SYS.Python
{
    Return ..%class
}

/// Get Classname
Method GetClassname() As %String
{
    Return ..%classname
}

/// Get Classname
Method GetModule() As %String
{
    Return ..%module
}

Method %OnNew(pConfigName As %String) As %Status
{
    $$$ThrowOnError(..Connect())
	Quit $method($this,"initConfig",.pConfigName) ; call subclass
}

Method OnInit() As %Status
{
    set tSC = $$$OK
    try {
        
        do ..%class."_dispatch_on_init"($this)
    } catch ex {
        set tSC = ex.AsStatus()
    }
    quit tSC
}

ClassMethod SetPythonPath(pClasspaths)
{
    set sys = ##class(%SYS.Python).Import("sys")

    for i=0:1:(sys.path."__len__"()-1) {
        Try {
            if sys.path."__getitem__"(i) = pClasspaths {
                do sys.path."__delitem__"(i)
            }
        }
        Catch ex {
            // do nothing
        }

    }
    do sys.path.insert(0, pClasspaths)
}

Method Connect() As %Status
{
    set tSC = $$$OK
    try {
        
        set container = $this
        
        //set classpass
        if ..%classpaths '="" {
            set delimiter = $s($system.Version.GetOS()="Windows":";",1:":")
            set extraClasspaths = $tr(container.%classpaths,delimiter,"|")
            for i=1:1:$l(extraClasspaths,"|") {
                set onePath = $p(extraClasspaths,"|",i)
                set onePath = ##class(%File).NormalizeDirectory(onePath)
                do ..SetPythonPath(onePath)
            }
        }
        if $isObject(..%class)=0 {
            set importlib = ##class(%SYS.Python).Import("importlib")
            set builtins = ##class(%SYS.Python).Import("builtins")
            set module = importlib."import_module"(..%module)
            set class = builtins.getattr(module, ..%classname)
            set ..%class = class."__new__"(class)
        }
        ;
        if ..%Extends("IOP.InboundAdapter") || ..%Extends("IOP.OutboundAdapter") {
            do ..%class."_set_iris_handles"($this,..BusinessHost)
        } elseif $this.%Extends("IOP.BusinessProcess") {
            do ..%class."_set_iris_handles"($this,$$$NULLOREF)
        } else {
            do ..%class."_set_iris_handles"($this,..Adapter)
        }
        ;
        do ..SetPropertyValues()
        ;
        try {
            do ..%class."_dispatch_on_connected"($this)
        } catch ex {
            $$$LOGWARNING(ex.DisplayString())
        }
        ;
    } catch ex {
        set msg = $System.Status.GetOneStatusText(ex.AsStatus(),1)
        set tSC = $$$ERROR($$$EnsErrGeneral,msg)
    }
    quit tSC
}

Method OnTearDown() As %Status
{
    set tSC = $$$OK
    if $isObject(..%class) {
        try {
            do ..%class."_dispatch_on_tear_down"()
        } catch ex {
            set tSC = ex.AsStatus()
        }
    }
    quit tSC
}

Method SetPropertyValues()
{
    set remoteSettings = $tr(..%settings,$c(13))
    for i=1:1:$l(remoteSettings,$c(10)) {
        set oneLine = $p(remoteSettings,$c(10),i)
        set property = $p(oneLine,"=",1) continue:property=""
        set value = $p(oneLine,"=",2,*)
        try {
            set $property(..%class,property) = value
        } catch ex {
            $$$LOGWARNING(ex.DisplayString())
        }
    }
    quit
}

Method dispatchSendRequestSync(
	pTarget,
	pRequest,
	timeout,
	pDescription) As %String
{
    set tSC = ..SendRequestSync(pTarget,pRequest,.objResponse,timeout,pDescription)
    if $$$ISERR(tSC) throw ##class(%Exception.StatusException).CreateFromStatus(tSC)
    quit $g(objResponse)
}

Method dispatchSendRequestSyncMultiple(
	pCallStructList As %List,
	pTimeout As %Numeric = -1) As %List
{
    set builtins = ##class(%SYS.Python).Import("builtins")
    // Convert %List to multidimensional array
    set tCallStructList=builtins.len(pCallStructList)
    for i=0:1:builtins.len(pCallStructList)-1 {
        set tCallStructList(i+1) = pCallStructList."__getitem__"(i)
    }

    set tSC = ..SendRequestSyncMultiple(.tCallStructList,pTimeout)
    if $$$ISERR(tSC) throw ##class(%Exception.StatusException).CreateFromStatus(tSC)

    // Convert multidimensional array to Python list
    set tResponseList = builtins.list()
    
    for i=1:1:tCallStructList {
        do tResponseList.append(tCallStructList(i))
    }
    quit tResponseList
}

Method dispatchSendRequestAsync(
	pTarget,
	pRequest,
	pDescription)
{
    set tSC = ..SendRequestAsync(pTarget,pRequest,pDescription)
    if $$$ISERR(tSC) throw ##class(%Exception.StatusException).CreateFromStatus(tSC)
    quit
}

ClassMethod OnGetConnections(
	Output pArray As %String,
	pItem As Ens.Config.Item)
{
    // finds any settings of type Ens.DataType.ConfigName
    Try {
        do ..GetPropertyConnections(.pArray,pItem)
    }
    Catch ex {
    }

    // Get settings
    do pItem.GetModifiedSetting("%classpaths", .tClasspaths)
    do pItem.GetModifiedSetting("%classname", .tClassname)
    do pItem.GetModifiedSetting("%module", .tModule)

    // try to instantiate class
    if tClasspaths '="" {
            set sys = ##class(%SYS.Python).Import("sys")
            set delimiter = $s($system.Version.GetOS()="Windows":";",1:":")
            set extraClasspaths = $tr(tClasspaths,delimiter,"|")
            for i=1:1:$l(extraClasspaths,"|") {
                set onePath = $p(extraClasspaths,"|",i)
                set onePath = ##class(%File).NormalizeDirectory(onePath)
                if onePath?1"$$IRISHOME"1P.E set onePath = $e($system.Util.InstallDirectory(),1,*-1)_$e(onePath,11,*)
                if onePath'="" do sys.path.append(onePath)
            }
    }
    set importlib = ##class(%SYS.Python).Import("importlib")
    set builtins = ##class(%SYS.Python).Import("builtins")
    set module = importlib."import_module"(tModule)
    set class = builtins.getattr(module, tClassname)
    set tClass = class."__new__"(class)

    set tPythonList = tClass."on_get_connections"()
    set tPythonListLen = tPythonList."__len__"()
    for i=0:1:(tPythonListLen-1) {
        set tPythonItem = tPythonList."__getitem__"(i)
        set pArray(tPythonItem) = ""
		#; set ^AALog(pItem.Name,tPythonItem) = ""
    }

    quit
}

Method dispatchSendRequestAsyncNG(
	pTarget,
	pRequest,
	pTimeout,
	pDescription,
	ByRef pMessageHeaderId,
	ByRef pQueueName,
	ByRef pEndTime) As %String
{
	set tSC=$$$OK, tResponse=$$$NULLOREF
	try {

        set tTargetDispatchName=pTarget
        set tTargetConfigName=$get($$$DispatchNameToConfigName(pTarget))
        if tTargetConfigName="" set tSC=$$$EnsError($$$EnsErrBusinessDispatchNameNotRegistered,tTargetDispatchName) quit
        set tTargetBusinessClass = $$$ConfigClassName(tTargetConfigName)
        set tINVOCATION=$classmethod(tTargetBusinessClass,"%GetParameter","INVOCATION")
        if (tINVOCATION'="Queue")&&(tINVOCATION'="InProc") set tSC=$$$ERROR($$$EnsErrParameterInvocationInvalid,tTargetBusinessClass) quit

        quit:$$$ISERR(tSC)
        ;
        set tStartTime=$zh
        set:pTimeout'=-1 tEndTime=$zh+pTimeout

        if tINVOCATION="InProc" {
            set tTimeout=$s(pTimeout=-1:-1,1:tEndTime-$zh)
            if (pTimeout'=-1)&&(tTimeout<0) quit
            set tSC=..SendRequestSync(tTargetConfigName,pRequest,.tResponse,tTimeout,pDescription)
            return tResponse
        } elseif tINVOCATION="Queue" {
            Set tSessionId=..%SessionId
            Set tSuperSession = ..%SuperSession
            Set tSC = ##class(Ens.MessageHeader).NewRequestMessage(.tRequestHeader,pRequest,.tSessionId,.tSuperSession) quit:$$$ISERR(tSC)
            Set ..%SessionId=tSessionId
            Set ..%SuperSession=tSuperSession
            Set tRequestHeader.SourceConfigName = ..%ConfigName
            Set tRequestHeader.TargetConfigName = tTargetConfigName
            Set tRequestHeader.SourceBusinessType = $$$ConfigBusinessType($$$DispatchNameToConfigName(..%ConfigName))
            Set tRequestHeader.TargetBusinessType = $$$ConfigBusinessType($$$DispatchNameToConfigName(tTargetConfigName))
            Set tRequestHeader.TargetQueueName = $$$getConfigQueueName($$$DispatchNameToConfigName(tTargetConfigName),..%SessionId)
            Set tRequestHeader.ReturnQueueName = $$$queueSyncCallQueueName
            Set tRequestHeader.BusinessProcessId = ""
            Set tRequestHeader.Priority = $$$eMessagePriorityAsync
            Set tRequestHeader.Description = pDescription
            Set tSC = ##class(Ens.Queue).Create($$$queueSyncCallQueueName) quit:$$$ISERR(tSC)
            Set tSC = ##class(Ens.Queue).EnQueue(tRequestHeader) quit:$$$ISERR(tSC)
            Set pMessageHeaderId = tRequestHeader.MessageId()
            Set pQueueName = $$$queueSyncCallQueueName
            Set:(pTimeout'=-1) pEndTime = tEndTime
        }
	}
	catch {
		set tSC = $$$EnsSystemError
	}
	quit tSC
}

Method dispatchIsRequestDone(
	pTimeout,
	pEndTime,
	pQueueName,
	pMessageHeaderId,
	ByRef pResponse) As %Status
{

    set tSC=$$$OK
    try {
        set tTimeout=$s(pTimeout=-1:-1,1:pEndTime-$zh)

        set tSC = ##class(Ens.Queue).DeQueue($$$queueSyncCallQueueName,.tResponseHeader,tTimeout,.tIsTimedOut,0) Quit:$$$ISERR(tSC)

        quit:$IsObject(tResponseHeader)=0

        set tFound = $select(tResponseHeader.CorrespondingMessageId: pMessageHeaderId=tResponseHeader.CorrespondingMessageId, 1: 0)
        if tFound=0 {

            set tSC = ##class(Ens.Queue).EnQueue(tResponseHeader)
            Kill $$$EnsActiveMessage($$$SystemName_":"_$Job)
        }
        else {

            if tIsTimedOut || ((pTimeout'=-1)&&(tTimeout<0)) {

                do tResponseHeader.SetStatus($$$eMessageStatusDiscarded)
                return $$$ERROR($$$EnsErrFailureTimeout, tTimeout, $$$StatusDisplayString(tSC), $$$CurrentClass)
            }
            if tResponseHeader.IsError {

                do tResponseHeader.SetStatus($$$eMessageStatusCompleted)
                return $$$EnsError($$$EnsErrGeneral,"Error message received: "_tResponseHeader.ErrorText)
                
            }
            if tResponseHeader.MessageBodyClassName'="" {

                set tResponse = $classmethod(tResponseHeader.MessageBodyClassName,"%OpenId",tResponseHeader.MessageBodyId,,.tSC)
                if '$IsObject(tResponse) return $$$EnsError($$$EnsErrGeneral,"Could not open MessageBody "_tResponseHeader.MessageBodyId_" for MessageHeader #"_tResponseHeader.%Id()_" with body class "_tResponseHeader.MessageBodyClassName_":"_$$$StatusDisplayString(tSC)) 
            } else {

                set tResponse=$$$NULLOREF
            }
            set pResponse=tResponse
            do tResponseHeader.SetStatus($$$eMessageStatusCompleted)
            set tSC = 2

        }
    }
	catch ex {
		set tSC = ex.AsStatus()
	}
	quit tSC
}

}


\begin{document}

\maketitle

\plt{Reflection is an important component of getting the full benefits from a
learning experience.  Besides the intrinsic benefits of reflection, this
document will be used to help the TAs grade how well your team responded to
feedback.  Therefore, traceability between Revision 0 and Revision 1 is and
important part of the reflection exercise.  In addition, several CEAB (Canadian
Engineering Accreditation Board) Learning Outcomes (LOs) will be assessed based
on your reflections.}

\section{Changes in Response to Feedback}

\plt{Summarize the changes made over the course of the project in response to
feedback from TAs, the instructor, teammates, other teams, the project
supervisor (if present), and from user testers.}

\plt{For those teams with an external supervisor, please highlight how the feedback 
from the supervisor shaped your project.  In particular, you should highlight the 
supervisor's response to your Rev 0 demonstration to them.}

\plt{Version control can make the summary relatively easy, if you used issues
and meaningful commits.  If you feedback is in an issue, and you responded in
the issue tracker, you can point to the issue as part of explaining your
changes.  If addressing the issue required changes to code or documentation, you
can point to the specific commit that made the changes.  Although the links are
helpful for the details, you should include a label for each item of feedback so
that the reader has an idea of what each item is about without the need to click
on everything to find out.}

\plt{If you were not organized with your commits, traceability between feedback
and commits will not be feasible to capture after the fact.  You will instead
need to spend time writing down a summary of the changes made in response to
each item of feedback.}

\plt{You should address EVERY item of feedback.  A table or itemized list is
recommended.  You should record every item of feedback, along with the source of
that feedback and the change you made in response to that feedback.  The
response can be a change to your documentation, code, or development process.
The response can also be the reason why no changes were made in response to the
feedback.  To make this information manageable, you will record the feedback and
response separately for each deliverable in the sections that follow.}

\plt{If the feedback is general or incomplete, the TA (or instructor) will not
be able to grade your response to feedback.  In that case your grade on this
document, and likely the Revision 1 versions of the other documents will be
low.} 

\subsection{SRS and Hazard Analysis}

\subsection{Design and Design Documentation}

\subsection{VnV Plan and Report}

\section{Challenge Level and Extras}

\subsection{Challenge Level}

\plt{State the challenge level (advanced, general, basic) for your project.  Your challenge level should exactly match what is included in your problem statement.  This should be the challenge level agreed on between you and the course instructor.}

\subsection{Extras}

\plt{Summarize the extras (if any) that were tackled by this project.  Extras
can include usability testing, code walkthroughs, user documentation, formal
proof, GenderMag personas, Design Thinking, etc.  Extras should have already
been approved by the course instructor as included in your problem statement.}

\section{Design Iteration (LO11 (PrototypeIterate))}

\plt{Explain how you arrived at your final design and implementation.  How did
the design evolve from the first version to the final version?} 

\plt{Don't just say what you changed, say why you changed it.  The needs of the
client should be part of the explanation.  For example, if you made changes in
response to usability testing, explain what the testing found and what changes
it led to.}

\section{Design Decisions (LO12)}

\plt{Reflect and justify your design decisions.  How did limitations,
 assumptions, and constraints influence your decisions?  Discuss each of these
 separately.}

\section{Economic Considerations (LO23)}

\plt{Is there a market for your product? What would be involved in marketing your 
product? What is your estimate of the cost to produce a version that you could 
sell?  What would you charge for your product?  How many units would you have to 
sell to make money? If your product isn't something that would be sold, like an 
open source project, how would you go about attracting users?  How many potential 
users currently exist?}

\section{Reflection on Project Management (LO24)}

\plt{This question focuses on processes and tools used for project management.}

\subsection{How Does Your Project Management Compare to Your Development Plan}

\plt{Did you follow your Development plan, with respect to the team meeting plan, 
team communication plan, team member roles and workflow plan.  Did you use the 
technology you planned on using?}

\subsection{What Went Well?}

\plt{What went well for your project management in terms of processes and 
technology?}

\subsection{What Went Wrong?}

\plt{What went wrong in terms of processes and technology?}

\subsection{What Would you Do Differently Next Time?}

\plt{What will you do differently for your next project?}

\section{Reflection on Capstone}

\plt{This question focuses on what you learned during the course of the capstone project.}

\subsection{Which Courses Were Relevant}

\plt{Which of the courses you have taken were relevant for the capstone project?}

\subsection{Knowledge/Skills Outside of Courses}

\plt{What skills/knowledge did you need to acquire for your capstone project
that was outside of the courses you took?}

\end{document}